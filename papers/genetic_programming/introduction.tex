\subsection{Introduction}
We use a supervised ML algorithm called genetic programming(GP)\cite{koza_genetic}. A detailed introduction to this method is beyond the scope of the paper, so we only present essential information pertaining to our work.

GP is evolutionary computation that employs naturally occurring genetic operations, a fitness function, and multiple generations of
Darwinian evolution to resolve a user-defined task \cite{GP_book}. GP can be used to discover a mathematical relationship between the input variables, also called features, in data (regression) or group data into categories (classification). Similar to biological natural selection, GP learns how well the program functions by comparing the output of its multivariate expression to a \emph{fitness score}. Programs with high fitness score are most likely, but not certain, to propagate to next generation. With subsequent generations, programs are more likely to get better at solving the task at hand \cite{Kai_thesis}. 

GP multivariate expressions are classically represented as a syntax tree, where the trees have a root (top center), nodes (mathematical operators), and leaves (operands). Operators can be arithmetic, trigonometric, boolean, etc. and the operands are place-holder variables related to the problem. The tree depth can be varied aiding in evolution of more complex multivariate expression. GP intitially generates a stochastic poplulation of trees, computes fitness scores and randomly selects trees for comparison. The lead scoring trees are selected and  one of the genetic operators (reproduction, mutation, crossover) is randomly applied to carry them forward to next generation. The process repeats until user-defined  conditions are met. The run-time parameters like intial population size, tree depth, tournament size for competing trees, number of generations, and termination criterion can be tuned for better algorithmic performance. %\cite{make sure from where!}%

The unique aspect of GP algorithm used here is the transparency of its internal workings. Unlike many black box ML algorithms, the evolutionary process can be reviewed at each step which might be quite important to many researchers.\cite{Cavaglia_2020}

For our analysis, we used an open source python code, Karoo GP \cite{KarooGP}. Karoo GP package is scalable, with multicore and GPU support enabled by the TensorFlow library and has capacity to work with very large datasets. The latest version of Karoo GP can be found in PyPI \cite{KarooPYPI}.   

