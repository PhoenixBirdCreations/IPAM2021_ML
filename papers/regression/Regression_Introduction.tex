\section{Introduction}
% mention BNS Fermi EM observation?
\lorena{More than seven years ago, the first gravitational wave (GW) from a compact
binary coalescence (CBC) was detected by the Laser Interferometer 
Gravitational-wave Observatory (LIGO)~\cite{LIGOScientific:2016aoc}. Ever 
since that first detection, the LIGO-Virgo-KAGRA (LVK) detector network has 
brought us to a total of more than 90 confident GW 
detections~\cite{LIGOScientific:2021djpq}. The sources of these detections 
originate from binary black hole (BBH) systems, binary neutron star (BNS) 
systems, and neutron star-black hole (NSBH) binaries. With the increased 
detector sensitivity in the fourth observing run (O4), the LVK is expected to 
detect as many as one GW signal per day~\cite{KAGRA:2013rdx}. This will 
require a quick turn-around time for analyses and results, prompting an
increase in automation of searches, data quality algorithms, and parameter
estimation (PE) pipelines.}

\lorena{In order to rapidly estimate the intrinsic and extrinsic parameters of CBC 
signals,the LVK detection pipelines use a technique called matched-filtering. 
This method takes the observed signal and matches it with a best-fit waveform
derived from general relativity. However, more accurate results on the source
parameters necessitates a follow-up with PE algorithms. These algorithms are 
robust but can take hours or days to run. This can be problematic when the 
signal observed comes from a system that radiates energy in the 
electromagnetic (EM) spectrum. Recent improvements in Bayesian techniques for 
PE have reduced the time needed to estimate the source parameters of a GW
detection~\cite{Ashton:2021anp,Wofford:2022ykb} and proposed machine learning 
(ML) algorithms promise to further reduce this latency by several orders of 
magnitude~\cite{Gabbard:2019rde}. Current online results using 
Bilby~\cite{Ashton:2018jfp} and RapidPE-RIFT~\cite{Lange:2018pyp} have 
decreased the time required for PE in low latency to be $\sim 10$ minutes. This 
result is obtained by suitably narrowing the PE priors and choosing aggresive 
sample settings. Nevertheless, there is still much to gain by reducing the 
latency of online PE even more. During O4, the LVK is planning to issue 
preliminary alerts for candidate detections seconds after the merger, and in
some cases, an early warning alert with information on sky localization may be 
issued~\cite{LVPublicAlerts}. Providing a higher level of accuracy at low 
latency will allow us to better estimate the time and peak luminosity of the 
afterglow emission in the case of a coincident EM signal.}
%mention our astronomy partners? 

%Introduce our work
\lorena{In this work, we present a novel approach to low latency PE using ML alongside 
the output of detection pipelines.}


%In our introduction we want to start from big picture (LIGO, ML, pipelines)
%and narrow down to what work we are presenting here and why it is important. We
%also describe in which ways it is novel and how it compares to previous works
%like in~\cite{Chatterjee:2019avs}. We may also want to cite~\cite{Sachdev:2020lfd}

\lorena{This paper is organized as follows.  In Sec. X we describe how the data is
prepared in order to be fed into GPR and NN.}
