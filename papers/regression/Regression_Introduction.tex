\section{Introduction}
%~\cite{Chatterjee:2019avs}. We may also want to cite~\cite{Sachdev:2020lfd}

% mention BNS Fermi EM observation? Could be added in second paragraph?
\lorena{More than seven years ago, the first gravitational wave (GW) from a compact
binary coalescence (CBC) was detected by the Laser Interferometer 
Gravitational-wave Observatory (LIGO)~\cite{LIGOScientific:2016aoc}. Ever 
since that first detection, the LIGO-Virgo-KAGRA (LVK) detector network has 
brought us to a total of more than 90 confident GW 
detections~\cite{LIGOScientific:2021djp}. The sources of these detections 
originate from binary black hole (BBH) systems, binary neutron star (BNS) 
systems, and neutron star-black hole (NSBH) binaries. With increased detector 
sensitivity in the upcoming fourth observing run (O4), the LVK is expected to 
detect as many as one GW signal per day~\cite{KAGRA:2013rdx}. This will 
require a quick turn-around time for analyses and results, prompting an
increase in automation of searches, data quality algorithms, and parameter
estimation (PE) pipelines.}

\lorena{In order to rapidly estimate the intrinsic and extrinsic parameters of
CBC signals, such as the masses, spins, and inclination angles, the LVK 
low-latency detection pipelines use a technique called matched-filtering. 
This method matches the observed GW signal with a waveform derived from 
general relativity. The intrinsic paramters of this best-fit waveform are 
then given as possible true parameters of the binary system detected. However, 
these parameters may not be the actual parameters of the
astrophysical system due in part to waveform systematics but mainly due to
detector noise. Moreover, a single event, or injection in our case, may ring
multiple triggers corresponding to different templates. This may add a layer of
complication. Therefore, for a detection a slower but more accurate PE follow-up
is needed. These PE algorithms are robust but can take hours or days to run. 
This can be problematic when the signal observed comes from a system that 
radiates energy in the electromagnetic (EM) spectrum. Recent improvements in 
Bayesian techniques for PE have reduced the time needed to estimate the source 
parameters of a GW detection~\cite{Ashton:2021anp,Wofford:2022ykb} and proposed 
machine learning (ML) algorithms promise to further reduce this latency by 
several orders of magnitude~\cite{Gabbard:2019rde}. Current online results 
using Bilby~\cite{Ashton:2018jfp} and RapidPE-RIFT~\cite{Lange:2018pyp} have 
decreased the time required for PE in low latency to be $\sim 10$ minutes. This 
result is obtained by suitably narrowing the PE priors and choosing aggresive 
sample settings. Nevertheless, there is still much to gain by reducing the 
latency of online PE further. During O4, the LVK is planning to issue 
preliminary alerts for candidate detections seconds after the merger, and in
some cases, an early warning alert with information on sky localization may be 
issued before~\cite{LVPublicAlerts}. Providing a higher level of accuracy at low 
latency will allow us to better estimate the time and peak luminosity of the 
afterglow emission in the case of a coincident EM signal. It will give the LVK
astronomy partners an observational head-start.}

\lorena{Although ML regression analysis has been recently applied to detector
characterization and data analysis~\cite{Yu:2021swq,Ormiston:2020ele,
Moore:2015sza,Williams:2019vub,Walker:2018ylg}, in this paper we present its 
first application in low latency PE. Specifically, we demonstrate that by using 
Gaussian process regression (GPR) and neural networks (NNs) we can significantly 
improve pipeline estimates of the masses and spin magnitudes of CBC detections.
Once trained, results can be obtained on timescales of ms per event, making
regression a powerful tool in low-latency. This allows us to bridge the gap 
between the speed and accuracy trade-off.}
