\subsection{Gaussian Process Regression (GPR)}

\lorena{Gaussian process regression is a sophisticated mathematical tool to find joint
Gaussian distributions between system data and the predictions. This makes it a
particularly powerful in interpolating between data sets and estimating
errors. Throughout this paper, we use \texttt{PyTorch}'s GPR module. We have to
make certain choices for our model such as the kernel function, loss function,
and the number of iterations.}

\lorena{A kernel funciton is a covariance function use to measure the similarity
between two data points. Our choice of kernel for training is the radial 
basis function (RBF) which is a squared exponential function given by}
\begin{equation}
K(X_1, X_2) = e^{-\frac{||X_1-X_2||^2}{2 \sigma^2}},
\end{equation}
\lorena{where $X_1$ and $X_2$ are input data points and $\sigma$ is the variance, 
or lengthscale parameter. This kernel is chosen for its accuracy and speed. 
The predictions of the O2 dataset are not strongly depend on the choice of
kernel. Apart from its popularity due to its versatility, the RBF kernel seems
to perform slightly better than other kernels and without a large loss in
speed. The loss function we use is negative marginal log-likelihood and we
train for 20 iterations. The Adams optimizer is used to find the optimal 
hyperparameters with a learning rate of $0.1$. This combination of learning 
rate and iterations allows us to decrease the loss function at a quick rate 
without a decrease in accuracy.}
