\section{Binary System Data}
\label{sec:data}

\lorena{The dataset represents over $203,000$ binary systems comprised
of BBHs, BNSs, and NSBHs. The injections of these systems come from the Mock
Data Challenge (MDC) conducted during LIGO's second observing run (O2) and are
recovered using \texttt{GstLAL}, a low-latency detection pipeline. We divide
this dataset into training and testing sets consisting of $\sim 142,000$ and
$\sim 61,000$ systems, respectively. Regression is applied to four features, 
namely the initial masses $m_1$ and $m_2$ and the initial spin magnitudes 
$\chi_1$ and $\chi_2$. The training data has masses that span 
$m_1 = [1.01, 119]$ and $m_2 =  [0.81, 81]$ and spin components ranging 
$\chi_{1,2} =  [-0.99, 0.99]$, as shown in Fig.~\ref{parameter_space}. For 
more information on the O2 MDC data mass and spin distributions, refer to Table 
I of~\cite{Chatterjee:2019avs}}.

\begin{figure}
	\centering
	\includegraphics[width=0.45\textwidth]{training_parameter_space.png}
    \caption{The $m_1$-$m_2$ (left) and $\chi_1$-$\chi_2$ (right) parameter space 
		    of injections from the O2 MDC.
            }
	\label{parameter_space}
\end{figure}

\lorena{ML algorithms require the data to be standardized, i.e. have a zero
mean and unit variance. However, before standardizing the data we take an 
additional step to ensure astrophysical values from the predictions. In other 
words, a prediction without proper data conditioning can result in negative
masses and spin magnitudes greater than one. To avoid this, we first map the 
data to be in the $(0,1)$ range by solving a simple, linear system of equations 
separately for the masses and the spins. We, then, transform our data using a 
sigmoid function so that the values map between $(-\infty,+\infty)$. It
is only at this point that the data is ready to be standardized and fed into
the GPR and NN algorithms. Once the testing data predictions are completed, 
the reverse process is applied, i.e., the data is scaled back from 
standardization, exponentiated, and mapped back from $(0,1)$ to astrophysical 
values.}
