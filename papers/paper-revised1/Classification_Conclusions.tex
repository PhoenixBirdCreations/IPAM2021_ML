\section{Conclusions\label{conclusions}}

In this paper, we have presented a new scheme for real-time classification of \ac{GW} \ac{CBC} signals detected by the \ac{LVK} detectors. The method uses the output of the \ac{LVK}
low-latency pipelines to identify whether the GW source progenitor contains a \ac{NS} (\hasns) and a post-merger matter remnant is produced in the merger (\hasrem). Estimates of these
metrics are included in public alerts for candidate \ac{GW} events issued by the \ac{LVK}. Determining these metrics in low latency is crucial to enabling coincident \ac{MMA} observations
of GW and \ac{EM} signatures.

We have assessed the viability and measured the performance of two classifiers, \ac{KNN} and \ac{RF}, on two sets of real detector data augmented with synthetic GW injections of GW
signals that were generated for space-time volume sensitivity analyses of \ac{O2} \ac{LVK} \ac{GW} searches~\cite{Chatterjee:2019avs} and an \ac{MDC} real-time replay of \ac{O3}
data~\cite{Chaudhary:2023vec}. 


One important novel ingredient of the proposed scheme is the computation of Bayesian probabilities for \hasns\ and \hasrem. Until now, the information that has been passed to astronomers
in public alerts has been in the form of binary classification scores for these metrics. Here, we provide a method to compute \hasns\ and \hasrem\ as actual probabilities that the \ac{GW}
source includes a neutron star and post-merger matter remnant. Therefore, our scheme provides more direct and easily interpretable information to aid the community of astronomers in
deciding whether to follow up on \ac{GW} candidate events with \ac{EM} observatories.

To construct the Bayesian probabilities for \hasns\ and \hasrem, we train and test the classifiers on the \ac{O2} data set following the customary 70\% -- 30\% split between training and
testing data. After evaluating the performance of the classifiers with standard \ac{ROC} curves, we use the testing set to generate numerical Bayesian probability expressions for the
models. This minimizes potential bias that may result from the use of data sets with different properties while ensuring that the Bayesian fits are built with data that is independent of the data used for training the classifiers. The effectiveness of the Bayesian fits is then evaluated on fully independent data sets using the \ac{O3} set and real detections. 

\tocheck{
As shown in Appendix~\ref{app:comparison}, our \ac{RF} implementation outperforms the specific \ac{KNN} algorithm implementation currently utilized in the \ac{LVK} low-latency infrastructure, while our \ac{KNN} method performs similarly.} When tested on the \ac{O3} set, both algorithms improve on \hasns\ while underperforming on
\hasrem.  In this case, \ac{KNN} outperforms \ac{RF}, which exhibits more variation across different pipelines.  If only the injections recovered by GstLAL are considered in \ac{O3}, the
\ac{O3} results of both \ac{RF} and \ac{KNN} are consistent with \ac{O2}. \ac{RF}'s performance on \ac{O3} events recovered by other pipelines, on the other hand, is noticeably lower.
This appears to imply that \ac{RF} is less portable than \ac{KNN} across different data sets and pipelines. The different ways \ac{RF} and \ac{KNN} operate, as well as the different
characteristics of the sets, may explain their behavior.

The RF classifier is a decision tree-based classifier that sets decision rules by implementing specific cuts (conditions) on input features. The \ac{KNN} algorithm implements decision
rules by computing the nearest neighbors of input features in the parameter space for the data point of interest. \ac{RF} is designed to construct hard boundaries based on input
parameters, whereas the \ac{KNN} algorithm is designed to produce an outcome based on differences between features. As a result, the \ac{RF} algorithm's nature may make it more suitable
for classifying events with \hasns, which is based on a well-defined, hard boundary between positive and negative outcomes, such as the secondary mass value. To distinguish between
systems with zero and nonzero post-remnant matter in \hasrem, the algorithms must learn Foucart's fit from the recovered parameters. Foucart's formula is dependent on the \ac{EOS} under
consideration, as well as the pipeline that recovers the injection. Because \ac{RF} and \ac{KNN} are trained on injections that are only recovered by GstLAL, \ac{RF} is more affected than
\ac{KNN}. \tocheck{However, a comprehensive evaluation, involving datasets from new observing runs and diverse pipelines, would be needed to definitely compare the performance of the two algorithms.}

This work provides an improved scheme to implement Bayesian probabilities for \hasns\ and \hasrem\ classification of candidate events that would be straightforward to deploy in the
existing LVK infrastructure. Our method can also be easily extended to other properties of \ac{GW} signals that are being or may be released in low latency, such as \hasgap\ among other
data products. Other future extensions of this work include improving algorithm training and Bayesian fit estimation with updated data sets of simulated injections in \ac{LVK} \ac{O4}
data generated with different pipelines and with better coverage of the mass gap region than the \ac{O2} data set. It would also be worthwhile to investigate the use of additional ML
classifiers that could further improve the process's accuracy, reduce the need for computational resources, and decrease latency. Finally, a similar infrastructure could be designed and
deployed to aid in the rapid parameter estimation of pipeline outputs, but with a focus on feature regression rather than classification. This latter line of investigation will be
presented in a future publication.


