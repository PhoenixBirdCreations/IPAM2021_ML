% voodoo for arXiv scripts
\pdfoutput=1
\documentclass[aps,prd,twocolumn,superscriptaddress,preprintnumbers,floatfix,nofootinbib]{revtex4-2}

\usepackage{graphicx}
\usepackage{natbib} 
\usepackage{amsmath}
%\usepackage{mdwlist}
%\usepackage[caption=false]{subfig}
%\usepackage{siunitx}
%\usepackage{placeins}
\usepackage{color}
%\usepackage{standalone}
\usepackage{dcolumn}
\usepackage{tensor}
\usepackage{bm}
\usepackage{makecell}
%\usepackage{MnSymbol}
\usepackage[T1]{fontenc}
\usepackage[utf8]{inputenc}
\usepackage{microtype}
\usepackage{etoolbox}
\usepackage{amssymb}
\usepackage{mathrsfs}
\usepackage{accents}
\usepackage[normalem]{ulem}
\usepackage[table,dvipsnames]{xcolor}
\usepackage[colorlinks,urlcolor=NavyBlue,citecolor=NavyBlue,linkcolor=NavyBlue,pdfusetitle]{hyperref}
\usepackage[all]{hypcap}
\usepackage[inline,shortlabels]{enumitem}
\usepackage{braket}
\usepackage{booktabs}
\usepackage{ulem}

\usepackage{array}
\usepackage{diagbox}
\usepackage{color}
\usepackage{colortbl}
\usepackage{hhline}
\usepackage{multirow}

\graphicspath{{./}{figs/}}

\newcommand{\nn}{\nonumber}
\newcommand{\NR}{\text{NR}}
\newcommand{\mmt}[1]{\textcolor{cyan}{#1}}
\newcommand{\lorena}[1]{\textcolor{RoyalPurple}{#1}}
\newcommand{\dimitra}[1]{\textcolor{teal}{#1}}

%%%%%%%%%%%%%%%%%%%%%%%%%%%%%%%%%%%%%%%%%%%%%%%%%%%%%%%%%%%%
%% Author information
%%%%%%%%%%%%%%%%%%%%%%%%%%%%%%%%%%%%%%%%%%%%%%%%%%%%%%%%%%%%


%\usepackage{orcidlink}

\newcommand{\MST}{\affiliation{Institute of Multi-messenger Astrophysics and Cosmology \& Physics Department, 
		Missouri University of Science and Technology, Rolla, MO 65409, USA}}
\newcommand{\OleMiss}{\affiliation{Department of Physics and Astronomy,
		University of Mississippi, University, Mississippi 38677, USA}}
\newcommand{\Torino}{\affiliation{Dipartimento di Fisica,
		Università di Torino \& INFN Sezione di Torino, via P. Giuria 1, 10125 Torino, Italy}}
\newcommand{\Tubingen}{\affiliation{Theoretical Astrophysics Department, 
		Eberhard-Karls University of T\"{u}bingen, T\"{u}bingen 72076, Germany}}
\newcommand{\UAB}{\affiliation{Departament de Matem\`{a}tiques,
		Universitat Aut\`{o}noma de Barcelona, 08193 Bellaterra, Spain}}
\newcommand{\UV}{\affiliation{Departament d’Astronomia i Astrof\'{i}sica,
		Universitat de Val\`{e}ncia, Dr. Moliner 50, 46100, Burjassot (Val\`{e}ncia), Spain}}
		
\begin{document}

%Title of paper
\title{Work in Classification}

\author{Simone Albanesi}
%	\orcidlink{0000-0001-7345-4415}}
\email{simone.albanesi@edu.unito.it}
\Torino
\author{Marina Berbel}
%	\orcidlink{0000-0001-6345-1798}}
\email{mberbel@mat.uab.cat}
\UAB
\author{Marco Cavagli\`{a}}
%	\orcidlink{0000-0002-3835-6729}}
\email{cavagliam@mst.edu}
\MST
\author{Lorena \surname{Magaña Zertuche}}
%	\orcidlink{0000-0003-1888-9904}}
\email{lmaganaz@go.olemiss.edu}
\OleMiss
\author{Miquel Miravet-Ten\'{e}s}
%	\orcidlink{0000-0002-8766-1156}}
\email{miquel.miravet@uv.es}
\UV
\author{Dimitra Tseneklidou}
%	\orcidlink{0000-0003-2582-1705}}
\email{dimitra.tseneklidou@uni-tuebingen.de}
\Tubingen
\author{Yanyan Zheng}
%	\orcidlink{0000-0002-5432-1331}}
\email{zytfc@umsystem.edu}
\MST
%\author{Any other?}
%	\orcidlink{}}
%\email{}


% Because hyperref only gets the *last* author, we need to be explicit.
\hypersetup{pdfauthor={LastName et al.}}

\date{\today}

\begin{abstract}
  The detection of gravitational wave signals generated by compact binary coalescences that involve neutron stars can led to multi-messenger observations of such events, due to the electromagnetic counterpart that is produced after merger. A rapid identification of the binary elements is necessary to perform proper follow-up operations with electromagnetic telescopes. This identification is done by looking at the properties of the binary system, such as the masses and spins inferred by low latency pipelines. However, the resulting predictions are dominated by systematic errors that might lead to a wrong identification of the system. In this work, we compare two supervised machine learning algorithms, K-Nearest Neighbors and Random Forest, that success in reducing such errors. We also present a new way to compute the probabilities of having a neutron star in the system and of producing an electromagnetic counterpart with a Bayesian approach.
\end{abstract}


%\maketitle must follow title, authors, abstract, \pacs, and \keywords
\maketitle

%\tableofcontents

% body of paper here - Use proper  commands
% References should be done using the \cite, \ref, and \label commands

\section{Introduction}
In our introduction we want to start from big picture (LIGO, ML, pipelines)
and narrow down to what work we are presenting here and why it is important. We
also describe in which ways it is novel and how it compares to previous works
like in~\cite{Chatterjee:2019avs}. We may also want to cite~\cite{Sachdev:2020lfd}. 


The breakthrough observation of the binary neutron star (BNS) merger simultaneously in gravitational and electromagnetic (E/M) waves \cite{LIGOScientific:2017ync} marked a new era of multi-messenger astronomy including for first time gravitational waves (GWs). The event GW170817 gave answers to open questions, such as the origin and production of Gamma-ray bursts (*cite), the production of heavy elements during a merger of BNS (*cite), rule out Equations of States (*cite) and alternative theories of gravity (*cite), +?. Among the challenges that it brings is the early alert of the E/M telescopes. Some of the expected sources to produce an E/M counterpart are the coalescences of a neutron star with another neutron star (NSNS) or with a black hole (NSBH). In the case of NSNS: GRBs, neutron star. For NSBH: accretion disk --> short GRBs.

Numerical relativity simulations can provide accurate information about the coalesence of the binary system and its aftermath/remnant. However, these simulations are computationally expensive and require a lot of time, making them inapropriate to alert follow-up E/M searches.
- Mention empirical fits, Foucart?

In the previous observing runs of LIGO and Virgo GW observations the methods used were:
Since the first observing run (O1) LIGO and Virgo use the GstLAL-based inspiral pipeline (GstLAL) to perform matched-filtering in low latency in order to detect GWs from CBC.


- Real time inference: What was done in O2, O3 (Deep)

- Probabilities HasNS, HasRemnant

- What we suggest, Random Forest classification (what is it, what are the advantages)

In this work, we revisit and extend the problem of classification in low-latency. First, we propose the use of a combined probability of whether there is a NS in the event and whether there is a remnant. By doing so, we exclude the unphysical situation of having a larger probability of having a remnant than having a NS (like in *ref). In addition, we perform a thorough study of different algorithms, perform several tests and account for different EoSs. We povide a direct comparison between the methods we used, as well as the ones already implemented e.g. in Deep *ref. 

The structure of the paper is as follows. In Section 2 there is a detailed description of the data used. In Section 3, the classification methods, namely Random Forest and KNN are presented. 

%here why ligo wants to classify, alert astronomers, try to improve previous work (KNN) exploring more algo (RF and GP), but present also KNN for direct comparison


\subsection{Data set} \label{dataset}

We use a large data set $D$ of simulated \ac{BNS}, \ac{NSBH}, and \ac{BBH} events that was first used for the space-time volume sensitivity analysis of the \ac{LVK} GstLAL search
\todo{[ref needed?]} and later employed in Ref.~\cite{Chatterjee:2019avs}. This allows us to directly compare the performance of the various algorithms and the new labeling
scheme to the performance of the \ac{KNN} algorithm that is deployed in the current \ac{LVK} observing run. 

The simulated signals are coherently injected in two-detector data from the \ac{O2} \ac{LVK} observing run. The injection population is built with uniform/loguniform distribution of component masses whereas the component spins are aligned and injected according to isotropic distributions. Further details on the waveforms and injection parameters can be found in Ref.~\cite{Chatterjee:2019avs}. The data set $D$ includes approximately 200,000 injected signals that are recovered by the GstLAL pipeline with a \ac{FAR} $\le$ 1/month. The \ac{RF} and \ac{KNN} algorithms are trained and tested on the injected and recovered intrinsic source properties (primary and secondary masses and spins) and on the recovered \ac{SNR}. 


%for ML we need data, we use injections in O2. Usually the labels of NS and REM were put in this way. Then we do multilabel with that in this other way.
%In particular to classify the REM we use 23 eos, that ligo chooses to be candidates covering wide parameter space (REF). So we have an algo for every of them. Probabilities are finally combined with weighted average by bayes factor (ref).

%\section{Improving Binary Classification through ML}
\label{sec:ML_concept}
In this section we may want to expand on the idea of using classification itself.
We should mention which algorithms we attempted to use and why they did not
work. Then we can talk more in detail about the ones that did work in the
following subsections.
%improving? that's not explaining the algo. I would call the section "Algorithms and results" or something similar
\section{Classification Algorithms}
\subsection*{\mmt{K-Nearest Neighbors}}

\todo{This section needs to be rewritten and expanded. Also, citations must be added. It should start with a brief sentence of what KNN is, then how it works, then what is our implementation. See the GP section ofr inspiration.}

\mmt{The K-Nearest Neighbors (KNN) is a non-parametric supervised algorithm~\cite{Fix:1951,Cover:1967} that relies on the fact that similar things are near to each other. It captures the idea of similarity by computing the distance between points, and it can be applied in classification problems~\cite{Guo:2004}, being renamed as KNeighborClassifier.}

\mmt{The KNeighborClassifier algorithm works as follows: the algorithm computes the distance between each point of data and the points of a training dataset, with a chosen metric. Then, it sorts the points in ascending order vased on the distance to the testing point. By choosing the top $K$ neighbors (training points) from the sorted array, KNN assigns a label to the data point, which is the most frequent label of the chosen neighbors.}

\mmt{In this work, we are going to use the KNeighborClassifier implementation of scikit-learn~\cite{Pedregosa:2011}, an open-source Python code. The algorithm is initialized through different parameters that should be tuned in order to optimize its performance. The number of neighbors is fixed by the user, and its optimal value turns out to be $K \approx 2n+1$~\cite{Chatterjee:2019avs}, where $n$ is the number of features that each datapoint has. The distance between the different neighbors can be computed with different metrics -euclidean, manhattan, chebysev, mahalanobis-, and the prediction can be weighted by the inverse of the distance or all the points in each neighborhood can be weighted equally. To compute the nearest neighbors, several algorithms are available, such as the BallTree, KDTree or a brute-force search. These are the parameters that we will tune in this work in order to optimize the algorithm for our concrete problem. We will apply cross validation by computing the score, i.e. the fraction of correctly classified events, over a range of possible values for the parameters of the algorithm.} 


\subsection{Random Forest (RF)}

A RF is an ensemble of decision trees. One of its major strengths is that every train trains and classifies independently from the rest, while the RF classification joints all the results and assign as category the mode from the trees. The probability of belonging to a category is therefore straightforward, being the number of trees that chose it divided by the total number of trees. Notice that the training and evaluation of a RF can be accelerated by parallelization, as computations inside each tree are independent from the rest.

The training of a RF is usually done with bootstrap, a technique that assigns a random subset of the training dataset to each tree. This prevents overfitting as every individual classifier is not exposed to the same data, and encourages pattern recognition by studying the same data from different subsets. Every decision tree is composed by nodes, where data its splitted until the different categories are separated. At each node, a subset of the features of the data is selected along threshold values that maximize the information gain at the separation. The binary splitting at each node gives the tree its name, as it can be visualized as roots going deeper at separations.

We use the RK implementation in scikitlearn \citep{Pedregosa:2011}. The main hyperparameters to tune in this module are the number of trees, the maximum depth allowed and the information gain criteria used at splitting (two are offered). We have observed that the maximum number of features to be considered in a node can be kept fix as the square root of the total number of features. Given that the aiming of this work is to improve the current low latency classification, the model once trained can occupy a restricted amount of memory. Therefore before searching the optimum hyperparameters for our dataset, we restrict those which make heavier models: the number of trees and their depth. We set to 300 the maximum number of trees the forest may have, and 45 their possible maximum depth.

For the RF we use events with 5 features: the two masses, their corresponding spins and the SNR of the detection. In the tuning of the hyperparameters we measure the performance  by its score: the number of events correctly classified against the total number of events in the testing dataset, if threshold is taken as 0.5. As all categories are balanced, this approach is enough to roughly compare models. 





\section{Results}

In order to decide which method gives a better performance in classifying this kind of events, we can apply them over testing data and finally do a comparison between both. A way to see how data is classified we can construct histograms where the number of events that are classified with a label (\texttt{HasNS/HasRemnant}) \texttt{True} or \texttt{False} will change with a given threshold of the probability. For an algorithm with perfect performance, all the events with label \texttt{True (False)} should be at \textit{p}(\texttt{label}) = 1 (\textit{p}(\texttt{label}) = 0).

Another way to check the algorithm's performance is by building the so-called \textit{Receiver Operating Characteristic (ROC) Curve}. They show the variation of the true-positive rate (or efficiency) with the false-positive rate given a certain threshold for the probability. An algorithm with a proper performance will give a steeper ROC curve, or in other words, will have a higher eficiency with a lower false-positive rate.  


In the ROC curves that we will present in the following subsections, we highlight three reference EoS in color, from which we show results in more detail. We select BHF\_BBB2 because is the model that give the lowest maximum mass, MS1\_PP as the model with the bigger maximum mass, and we also include SLy because is the most accepted EoS for NS modeling (reference), and is the one that was used in the injections that are our dataset.

Here are the results for both methods:
\subsection{KNN Results}

%\mmt{We are only using 5 features, the independent variables: 
%\begin{equation*}
	%\big[m_1,m_2,\chi_1,\chi_2, \rm{SNR}\big]\,.
%\end{equation*}}

\subsubsection{\mmt{Has NS}}
\mmt{The metric we use to compute the distance between neighbors is the \textit{Manhattan} metric (or the Minkowski's $L1$ distance),  which is the distance between two points measured along axes at right angles. Having $p_1(x_1,y_1)$ and $p_2(x_2,y_2)$ the distance will be}
\begin{equation}
	d = |x_1-x_2|+|y_1-y_2|\,.
\end{equation}

\mmt{Moreover, the points are weighted uniformly.  After applying cross-validation,  we get that the optimal number of neighbors is $K_{\rm NS} = 10$, with a mean score $\rm{s_m} = 0:9718355224352762$ and a testing score  $\rm{s_t} = 0.9723828730478842$.} 

%\begin{figure}
%    \includegraphics[width = 0.4\textwidth]{CrossValK.pdf}
%    \caption{Score of our KNN model as a function of the number of neighbors. We are considering \textit{HasNS}.}
%    \label{fig:crossvalK}
%\end{figure}
    
\begin{figure}
    \includegraphics[width=0.45\textwidth]{figs/conf_matrix_NS.pdf}
    \caption{Confusion matrix for our model for \textit{HasNS}, using the independent recovered values. }
    \label{fig:confmat}
\end{figure}

\begin{figure}
    \includegraphics[width = 0.4\textwidth]{plot_fig4_chatt_spins.png}
  %   \includegraphics[width = 0.4\textwidth]{/Users/miquelmiravet/Projects/IPAM_LA/ML_group/IPAM2021_ML/algo/classy_KNN/PLOTS_KNN/NS_set/plots_miq/plot_fig4_chatt_snr.png}
    \caption{Probability of having a remnant as a function of the values of the masses. The different panels show the results for different spins. The solid red line depicts the threshold mass for $m_2$.}
    \label{fig:m1m2}
\end{figure}

\begin{figure}
	\includegraphics[width =0.4\textwidth]{ROCplot.pdf}
    \caption{Relation of the true and false positive rates as a function of the threshold applied to make the decision between having or not having a remnant. }
    \label{fig:roc}
\end{figure}

\mmt{In Fig. you can find how the mean score changes with the number of neighbors of the algorithm.  The confusion matrix appears in Fig.~\ref{fig:confmat}, the probability as a function of $m_1$ and $m_2$  is shown in Figs.~\ref{fig:m1m2}, and the true and false positive rates in terms of the threshold probability appear in Fig.~\ref{fig:roc}.}


 %plots and comments
\subsection{RF Results}

\begin{figure}
\centering
\includegraphics[width=0.9\textwidth]{/figs/BHF_BBB2_NShist}
\includegraphics[width=0.9\textwidth]{/figs/BHF_BBB2_REMhist}
\caption{\label{fig:RF_hist_BHFBBB2}}
\end{figure}
 %plots and comments
\subsection{Algorithm comparison}
%\section{Results}
%Here we talk about overall results and specifically from each algorithm in the
%subsections below.  \mmt{[MMT: Below I described how to check the performance of the algorithms. Maybe a table with all the scores/sensitivities/precisions from both %KNN and RF would be useful (already got it in a google doc)]}

%\mmt{To measure the performance of the classifiers we use some common statistical quantities.  The score is the number of correctly predicted events over the number %of total events (a perfect classifier has a score of 1).  It works best when there is an equal number of events for each label in the training set. It does not %consider the importance of misclassification, or that the training data can be biased towards one specific label.}

%\mmt{The mean score is computed by training the algorithm on the $90\%$ of the dataset and testing it on the remaining $10\%$, cycling the train/test combination over %the full dataset. To do that, we are going to use the training dataset, since it's the larger one.  In order to train and test the model and create the different %plots, we are going to use the training and testing files. }

%\mmt{Another useful quantity is the sensitivity. It is the ratio between the true positives and the sum of the true positives and false negatives.  It measures how %much the algorithm predicts \textit{true} (in our case it would be that the event has NS or has REM), when it is actually \textit{true}.Having a sensitivity equal to %1 would mean that our method predicts \textit{true} for every event. Therefore, a method with high sensitivity will barely miss true alarms. }

%\mmt{A quantity that measures how much you can trust a method when it predicts \textit{true} is the precision.  It is the ratio between the true positives and the %some of the true  and the false positives. A precision equal to 1 means that the method never predicts \textit{true} when it is actually \textit{false}. This means %that the method will never give false alarms. }

%\mmt{Finally, the F1 score $F1 = 2(\rm{precision \times sensitivity})/(\rm{precision+sensitivity})$ is a type of score that takes into consideration how precision and %sensitivity compensate each other. A perfect classifier would have a F1 score of 1.}

To compare quantitatively the results from RF and KNN we compute the true positive and false positive rate for several threshold values, for both HasNS and HasREM, for the three selected EoS. These are tables \ref{tab:TPbhf}, \ref{tab:TPms1} and \ref{tab:TPsly}. For HasNS the two algorithms perform similarly, with almost the same TP for all threshold values and accross EoSs, although the false positive is smaller always in the RF. For HasREM we obtain that RF performs better than KNN in every case, with not only a smaller false positive rate, but a greater true positive rate.

\begin{table}[]
\centering
\begin{tabular}{@{}c|cccc|cccc@{}}
\toprule
\multicolumn{1}{l|}{}          & \multicolumn{4}{c|}{Has NS}                       & \multicolumn{4}{c}{Has REM}                      \\ \midrule
                               & \multicolumn{2}{c}{RF} & \multicolumn{2}{c|}{KNN} & \multicolumn{2}{c}{RF} & \multicolumn{2}{c}{KNN} \\
\multicolumn{1}{l|}{Threshold} & TP         & FP        & TP          & FP         & TP         & FP        & TP         & FP         \\ \midrule
0.1                            & 0.999      & 0.107     &   0.999          &  0.156          & 0.998      & 0.011     &    0.992        &  0.051          \\
0.3                            & 0.998      & 0.068     &   0.996        &  0.117          & 0.993      & 0.005     &   0.974         &  0.017          \\
0.5                            & 0.994      & 0.042     &   0.991          &  0.088           & 0.985      & 0.003     &   0.937         &  0.006          \\
0.8                            & 0.967      & 0.014     &   0.966          & 0.043            & 0.957      & 0.001     &  0.845          &   0.001         \\ \bottomrule
\end{tabular}
\caption{BHF\_BB2}
\label{tab:TPbhf}
\end{table}


\begin{table}[]
\centering
\begin{tabular}{@{}c|cccc|cccc@{}}
\toprule
\multicolumn{1}{l|}{}          & \multicolumn{4}{c|}{Has NS}                       & \multicolumn{4}{c}{Has REM}                      \\ \midrule
                               & \multicolumn{2}{c}{RF} & \multicolumn{2}{c|}{KNN} & \multicolumn{2}{c}{RF} & \multicolumn{2}{c}{KNN} \\
\multicolumn{1}{l|}{Threshold} & TP         & FP        & TP          & FP         & TP         & FP        & TP         & FP         \\ \midrule
0.1                            & 1.000      & 0.114     &    0.999         & 0.138           & 0.999      & 0.023     &    0.995        &  0.103         \\
0.3                            & 0.998      & 0.065     &   0.995          & 0.097            & 0.996      & 0.010     &  0.983          & 0.044           \\
0.5                            & 0.994      & 0.036     &  0.989           & 0.068           & 0.990      & 0.004     &   0.961         & 0.019           \\
0.8                            & 0.968      & 0.011     &    0.967         & 0.031            & 0.967      & 0.001     &   0.899         &   0.004         \\ \bottomrule
\end{tabular}
\caption{MS1\_PP}
\label{tab:TPms1}
\end{table}

\begin{table}[]
\centering
\begin{tabular}{@{}c|cccc|cccc@{}}
\toprule
\multicolumn{1}{l|}{}          & \multicolumn{4}{c|}{Has NS}                       & \multicolumn{4}{c}{Has REM}                      \\ \midrule
                               & \multicolumn{2}{c}{RF} & \multicolumn{2}{c|}{KNN} & \multicolumn{2}{c}{RF} & \multicolumn{2}{c}{KNN} \\
\multicolumn{1}{l|}{Threshold} & TP         & FP        & TP          & FP         & TP         & FP        & TP         & FP         \\ \midrule
0.1                            & 1.000      & 0.107     &   0.999          &           0.155 & 0.998      & 0.013     &  0.992          & 0.059           \\
0.3                            & 0.999      & 0.064     &  0.996           &           0.112 & 0.993      & 0.005     & 0.974           & 0.020           \\
0.5                            & 0.994      & 0.038     &  0.990           &           0.084 & 0.986      & 0.003     & 0.940           &   0.007         \\
0.8                            & 0.965      & 0.012     &   0.965          &           0.040 & 0.958      & 0.001     & 0.848           &  0.001          \\ \bottomrule
\end{tabular}
\caption{SLy}
\label{tab:TPsly}
\end{table}

\begin{table}[]
\centering
\begin{tabular}{cccccccccccccc}
\hline
event ID & grace ID & $m_{1 {\rm rec}}$ & $m_{2{\rm rec}}$ & $\chi_{1{\rm rec}}$ & $\chi_{2{\rm rec}}$ & SNR & GWTC & pHasNS & pHasREM & pHasNS-RF & pHasREM-RF & pHasNS-KNN & pHasREM-KNN \\
\hline 
\hline
GW170823 &	G298936 &	59.126324 &	24.816019 &	-0.298205 &	0.898520 &	11.2960 &	1 & 0.000 & 0.000 &	0.000 &	0.000 & 0.000 &	0.000 \\
GW170817 &	G298048 &	1.527005 &	1.242296 &	-0.015902 &	-0.035747 &	14.4500 & 1 & 1.000 & 1.000 &	1.000 &	1.000 & 1.000 & 1.000 \\
GW170814 &	G297595 &	29.478622 &	24.901943 &	-0.568798 &	0.130793 &	16.1496 &	1 & 0.000 & 0.000 &	0.000 &	0.000 & 0.000 & 0.000 \\
GW170809 &	G296853 &	43.061466 &	30.084999 &	-0.120968 &	0.846714 & 	11.2619 &	1 & 0.002 &	0.000 &	0.002 & 0.000 & 0.000 & 0.000 \\
GW190408 &	G329243 &	32.132198 &	23.224018 &	0.279017 &	-0.647443 &	13.9286 &	2 & 0.000 &	0.000 	&	0.000 &	0.000 & 0.000 & 0.000 \\
GW190412 &	G329483 &	26.368868 &	12.002029 &	0.447732 &	-0.726978 &	18.2125 &	2 & 0.000 &	0.000 &	0.000 &	0.000 & 0.000 & 0.000 \\
GW190413-052954 &	G329577 &	45.688461 &	32.476582 &	-0.871232 &	0.141259 & 	9.0342 &	2 & 0.000 &	0.000 	 &	0.000  & 0.000 & 0.000 & 0.000 \\
GW190413-134308 &	G329615 &	73.181816 &	67.166664 &	-0.143886 &	0.241004 &	10.3324 &	2 & 0.000 &	0.000  & 0.000 &	0.000 & 0.000 & 0.000 \\
GW190421 &	G330300 &	57.254688 &	21.741816 &	-0.995402 &	-0.216579 &	9.9563 &	2 & 0.000 &	0.000 	&	0.000 &	0.000 & 0.000 & 0.000 \\
GW190425 &	G330564 &	2.266128 &	1.306907 &	0.088769 & -0.027512 & 	13.0987 &	2 & 1.000 &	0.995 &	1.000 &	0.999 & 1.000 & 1.000 \\
GW190426 &	G330687 &	7.590450 &	1.339249 &	0.067198 &	-0.007581 &	10.0590 &	2 & 1.000 &	0.009 &	0.996 & 0.009 & 1.000 & 0.000 \\
GW190503 &	G331315 &	55.742626 &	7.720621 &	-0.740169 &	-0.829573 &	11.9346 &	2 & 0.000 &	0.000 &	0.000 &	0.000 & 0.000 & 0.000 \\
GW190512 &	G332169 &	26.352001 &	16.432217 &	-0.598480 &	0.802231 &	12.3291 &	2 & 0.000 	&0.000 & 0.000 &	0.000 & 0.000 & 0.000 \\
GW190513 &	G332333 &	47.117332 &	23.614485 &	0.038690 &	0.145947 &	12.5778 &	2 & 0.000 &	0.000 &	0.000 &	0.000 & 0.000 & 0.000 \\
GW190517 &	G333132 &	48.227360 &	43.633629 &	0.653985 &	0.932955 &	11.1693 &	2 & 0.000 &	0.000 &	0.000 &	0.000 & 0.000 & 0.000 \\
GW190519 &	G333443 &	91.274368 &	37.288105 &	0.322839 &	0.147511 &	13.5088 &	2 & 0.000 &	0.000 &	0.000 &	0.000 & 0.000 & 0.000 \\
GW190521-074359 &	G333664 & 42.442223 &	33.601784 &	-0.697021 &	-0.155956 &	23.7673 &	2 & 0.000 &	0.000 &	0.000 &	0.000 & 0.000 & 0.000 \\
GW190602 &	G335015 & 99.914513 &	5.767400 &	-0.389662 &	-0.888722 &	11.9352 &	2 & 0.001 &	0.000 &	0.000 &	0.000 & 0.000 & 0.000 \\
GW190630 &	G337426 &	47.117332 &	23.614485  &	0.038690 &	0.145947 &	15.2795 &	2 & 0.000 &	0.000 	&	0.000 &	0.000 & 0.000 & 0.000 \\
GW190706 &	G337919 &	133.333330 &	37.090908 &	0.590816 &	0.969248 & 	12.4635 &	2 &	0.003 &	0.000 & 0.015 &	0.000 &	0.000 &	0.000 \\ 
GW190707 &	G337978 &	41.523457 &	4.112645 &	0.584189 &	0.589589 & 	12.5254 &	2 &	0.000 &	0.000 & 0.000 & 0.000 &	0.000 &	0.000 \\
GW190708 &	G338125 &	78.389603 &	5.766267 &	0.686623 &	0.813652 &	11.9672 &	2 &	0.000 &	0.000 & 0.000 &	0.000 &	0.000 &	0.000 \\
GW190720 &	G344653 &	17.278727 &	8.200110 &	-0.061079 &	0.762016 &	10.6910 &	2 &	0.002 &	0.000 & 0.004 &	0.000 &	0.000 &	0.000 \\
GW190727 &	G345173 &	47.362637 &	28.559050 &	-0.953421 &	0.030223 &	12.1147 &	2 &	0.005 &	0.000 & 0.011 &	0.000 &	0.000 &	0.000 \\
GW190728 &	G345315 &	21.592342 &	21.592342 &	0.459959 &	-0.600577 &	13.5572 &	2 &	0.001 &	0.000 & 0.000 &	0.000 &	0.000 &	0.000 \\
GW190814 &	G347305 &	20.846231 &	3.045587 &	-0.220753 &	0.386755 &	24.4970 & 	2 &	0.097 &	0.000 & 0.098 &	0.000 &	0.700 &	0.000 \\
GW190828-063405 &	G348500 &	52.002979 &	27.973516 &	-0.025977 &	0.329331 &	16.0000 &	2 &	0.000 &	0.000 & 0.002 &	0.000 &	0.000 & 0.000 \\
GW190828-065509 & 	G348519 &	30.383705 &	12.924967 &	-0.241386 &	0.694826 & 	10.9892 &	2 &	0.001 &	0.000 & 0.001 &	0.000 &	0.000 &	0.000 \\
GW190915 &	G350491 &	34.684841 &	32.121094 &	0.211614 &	-0.951023 &	13.0914 &	2 &	0.000 &	0.000 & 0.000 &	0.000 &	0.000 &	0.000 \\
GW190924 &	G351423 &	14.338296 &	4.125206 &	0.165658 &	0.590497 &	13.1556 &	2 &	0.054 &	0.000  & 0.037  & 0.000 &	0.070 &	0.000 \\
GW190930 &	G351993 &	36.524567 &	4.330317 &	0.674522 &	-0.918339 &	10.0846 &	2 &	0.004 &	0.000 & 0.000 &	 0.000 &	0.000 &	0.000 \\ 
GW191109 &	G354231 &	106.235710 & 10.805781 & 0.495254 &	0.278808 &	14.1976 &	3 &	0.000 &	0.000 & 0.000 &	0.000 &	0.000 & 	0.000 \\
GW191129 &	G355916 &	13.095535 &	7.427211 &	0.118102 &	0.101901 &	13.0631 &	3 &	0.002 &	0.000 & 0.004 & 	0.000 &	0.000 & 0.000 \\
GW191204-171526 &	G356500 &	16.022400 &	7.993040 &	0.111097 &	0.573796 &	16.9267 &	3 &	0.000 &	0.000 & 0.000 &	0.000	& 0.000 &	0.000 \\
GW191215 &	G357403 &	47.154507 &	17.760225 &	0.077376 &	0.273546 &	10.6819 &	3 &	0.000 &	0.000 & 0.000 &	0.000 &	0.000 &	0.000 \\
GW191216 &	G357490 &	34.448246 &	3.892543 &	0.613180 &	0.370414 &	18.6183 &	3 &	0.000 &	0.000 & 0.000 &	0.000 &	0.000 & 0.000 \\
GW191222 &	G358088 &	80.275490 &	11.620949 &	-0.734323 &	-0.322937 &	12.2304 &	3 &	0.000 &	0.000 & 0.000 &	0.000 &	0.000 & 0.000 \\
GW200112 &	G359994 &	50.928673 &	32.735661 &	0.752192 &	-0.974956 &	18.7937 &	3 &	0.000 &	0.000 & 0.000 &	0.000 &	0.000 & 	0.000 \\
GW200115 &	G360364 &	4.304630 &	2.084976 &	-0.652943 &	0.041389 &	11.4240 &	3 &	1.000 &	0.001 & 	0.997 &	0.006 &	1.000 &	0.000 \\
GW200129 &	G361581 &	47.739670 &	29.955526 &	0.831136 &	-0.983558 &	26.6116 &	3 &	0.000 &	0.000 & 0.000 &	0.000 &	0.000 &	0.000 \\
GW200219 &	G364596 &	61.230534 &	10.482111 &	-0.813236 &	-0.926772 &	10.5899 &	3 &	0.001 &	0.000 & 0.000 &	0.000 &	0.000 	& 0.000 \\
GW200224 &	G365371 &	86.812164 &	13.472847 &	0.009163 &	0.009163 &	16.9285 &	3 &	0.012 &	0.000 & 0.014 &	0.000 &	0.000 & 	0.000 \\
GW200225 &	G365427 &	20.317394 &	19.071180 &	0.386950 &	-0.901944 &	12.4941 &	3 &	0.001 &	0.000 & 0.001 &	0.000 &	0.000 &	0.000 \\ 
GW200302 &	G366190 &	47.138294 &	27.403164 &	0.361661 &	-0.600572 &	10.9044 &	3 &	0.000 &	0.000  & 0.000 &	0.000 &	0.000 & 0.000 \\
GW200311 &	G367788 &	34.684841 &	32.121094 &	0.211614 &	-0.951023 &	17.7958 &	3 &	0.000 &	0.000 & 0.000 &	0.000 & 0.000 & 0.000 \\   
GW200316 &	G368545 &	36.925163 &	4.943711 &	0.587454 &	-0.145408 &	10.2106 &	3 &	0.000 &	0.000  & 0.000 &	0.000 &	0.000 & 0.000 \\
GW200322 &	G369200 &	54.669945 &	15.419386 &	0.995742 &	0.862830 &	8.7530 &	3 &	0.000 &	0.000 & 0.012 &	0.000 &	0.000 &	0.000 \\


\hline
\end{tabular}
\caption{PROBAB TABLE REAL DATA}
\label{tab:real_data}
\end{table}

%comparing both, TP and FP and what do we think


\section{Conclusions\label{conclusions}}
Reiterate why what we did is important and how it improves current knowledge.


\section*{Acknowledgments}


We are very grateful to the authors of Ref.~\cite{Chatterjee:2019avs} for fruitful discussions and for sharing their work that helped us to directly compare
our methods to those used in \ac{LVK}'s \ac{O3} observing run. We would also like to thank the many other colleagues of the LIGO Scientific Collaboration and
the Virgo Collaboration who have provided invaluable help over the years.

This work is based upon work supported by the LIGO Laboratory which is a major facility fully funded by the National Science Foundation. We are grateful for
computational resources provided by the LIGO Laboratory and supported by the U.S.\ National Science Foundation Grants PHY-0757058 and PHY-0823459. L.M.Z.\ is partially
supported by the MSSGC Graduate Research Fellowship, awarded through the NASA Cooperative Agreement 80NSSC20M0101. M.C.\ and Y.Z.\ are partially supported by the U.S.\
National Science Foundation under award PHY-2011334. M.M.T.\ is supported by the Spanish Ministry of Universities through the Ph.D.\ grant No.\ FPU19/01750, by the
Spanish Agencia Estatal de Investigaci\'on (Grants No.\ PGC2018-095984-B-I00 and PID2021-125485NB-C21) and by the Generalitat Valenciana (Grant No.\ PROMETEO/2019/071).
M.B.\ is supported by the Spanish Agencia Estatal de Investigaci\'on (Grants No. PID2020-118236GB-I00). Part of this research was performed at the Institute of Pure and
Applied Mathematics (IPAM),  University of California Los-Angeles (UCLA). IPAM is partially supported by the National Science Foundation through award
DMS-1925919. We would like to thank IPAM and UCLA for their warm hospitality. 

Software for this analysis is written in \texttt{Python} 3.x \todo{add ref} and uses standard Open Source libraries and community-contributed modules from the Python Package Index (PyPI) repository \todo{add ref} including \texttt{numpy}, \texttt{scipy}, \texttt{pandas}, \texttt{matplotlib} \cite{Hunter:2007ouj}, and \texttt{sklearn}.

This manuscript has been assigned LIGO Document Control Center number LIGO-P23XXXXX.







%\def\bibsection{\section*{References}}
%\bibliographystyle{aps}
\bibliography{classification}

\end{document}

% Local Variables: 
% mode: latex
% End:
