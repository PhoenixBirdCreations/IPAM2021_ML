\section{Introduction\label{intro}}
%In our introduction we want to start from big picture (LIGO, ML, pipelines)
%and narrow down to what work we are presenting here and why it is important. We
%also describe in which ways it is novel and how it compares to previous works
%like in~\cite{Chatterjee:2019avs}. We may also want to cite~\cite{Sachdev:2020lfd}. 

The first detection of a \ac{GW} signal from a pair of coalescing black holes in 2015 and the first observation of a coalescing binary neutron system two years later
have established \ac{MMA} as a powerful tool for the exploration of the cosmos~\cite{LIGOScientific:2016aoc,LIGOScientific:2017vwq}. The third \ac{LVK} catalog of transient \ac{GW} signals \ac{GWTC3}~\cite{LIGOScientific:2021djp}
has shown that \ac{GW} astronomy has entered its mature phase, becoming a true observational branch of astronomy. In the next few years, \ac{MMA} will allow scientists
to explore in depth the origin and structure of black holes, neutron stars, and gamma-ray bursts; test general relativity; probe the fundamental nature of gravity; and
measure the evolution of the universe~\cite{LIGOScientific:2021sio,LIGOScientific:2021psn,LIGOScientific:2021aug} \todo{add reference to structure of BHs, NSs, and GRBs}. 

Improved \ac{GW} detector sensitivities will bring a wealth of new detections to achieve these science goals~\cite{LIGOScientific:2014pky,VIRGO:2014yos}. The rate
of detections in \ac{LVK}'s \ac{O4} is expected to be more than one per day, further increasing in the \ac{O5}~\cite{KAGRA:2013rdx}. Over the course of
these runs, the \ac{LVK} will analyze hundreds of \ac{BBH} and dozens of \ac{BNS} and \ac{NSBH} detections, several of which could be \ac{MMA} sources. Among the
challenges that this new phase of \ac{GW} astronomy brings is the necessity to coordinate the activities of \ac{EM} and \ac{GW} observatories on a very short time
scale. 

One of the most interesting areas of study in \ac{GW} astronomy is the physics of gravity-matter interaction in \ac{MMA} sources. Tidally disrupted matter in an
\ac{NSBH} system may form a high-temperature accretion disk around the \ac{BH} and trigger the creation of a prompt \ac{EM} emission in the form of a short \ac{GRB}. If
the system ejecta are unbound, r-process nucleosynthesis may lead to a \ac{KN} \cite{Lattimer:1974slx, Li:1998bw, Korobkin:2012uy, Barnes:2013wka, Tanaka:2013ana,
Kasen:2014toa}. These phenomena could also arise in \ac{BNS} post-mergers through the expulsion of neutron-rich material even when tidal forces are weak
\cite{LIGOScientific:2017ync, Arcavi:2017xiz, Coulter:2017wya, Kasliwal:2017ngb, Lipunov:2017dwd, DES:2017kbs, Tanvir:2017pws}. The presence of a post-merger matter
remnant, which results in an \ac{EM} signature or a prompt collapse, is a common factor in all of these scenarios. Determining the potential of a \ac{GW} source to
become an \ac{EM} emitter and enabling coincident observations of these systems by \ac{EM} and \ac{GW} observatories in real time are crucial for the success of
\ac{MMA}.

The \ac{LVK} employs different matched-filtering pipelines for low-latency \ac{GW} searches \cite{Sachdev:2020lfd,Nitz_2018,Adams_2016} \todo{references for pipelines
and add spiir}. These searches are based on discrete template banks of \ac{CBC} waveforms that provide, among other parameters, the component masses and the
dimensionless (anti-)aligned spins of the objects along the orbital angular momentum. These parameters can be used to determine the \emph{\ac{EM} properties} of
\ac{GW} candidates in low latency through empirical fits of \ac{NR} simulations \cite{Foucart:2012nc,Foucart:2018rjc}. Low-latency preliminary alerts of candidate
\ac{GW} events in the \ac{O3} included two \ac{EM}-property metrics identifying whether the \ac{CBC} system contains a \ac{NS}, \hasns, and a post-merger matter
remnant, \hasrem. Alerts with similar content will continue to be issued in \ac{O4} with a latency of the order of one second after the detection of candidate merger events.  Additionally, \ac{LVK}'s \ac{O4} alerts will include a measure of the \hasgap\ property, i.e., the likelihood that one of the source compact objects has a mass in the lower-mass gap region between \ac{NS}s and \ac{BH}s

Classification of \ac{GW} candidate events in real time poses several challenges as the desire for accuracy contrasts with the desire to issue the information as
quickly as possible. The approach taken in \ac{O3} was to use a supervised \ac{KNN} \cite{Pedregosa:2011ork} \ac{ML} algorithm with input from the detection pipelines
and \ac{EOS} models to generate independent \hasns\ and \hasrem\ binary classification scores \cite{Chatterjee:2019avs}. The \ac{KNN} model was trained on a broad set
of synthetic \ac{CBC} signals injected in real detector noise from the \ac{O2}. The advantage of this scheme relies in the capability of the method to handle the
statistical uncertainties of the parameters from the search pipelines and the non-stationarity of the detectors' power spectral density. This allowed for a marked
improvement in performance compared to the semi-analytic effective Fisher formalism method which was deployed in \ac{O2}.

In this work, we revisit the problem of real-time production of the \hasns\ and \hasrem\ metrics with the aim of further improving the latency and performance of the
\ac{ML}-based scheme. To this purpose, we explore a set of new algorithms and classification schemes. In our approach, we first design a scheme that allows for
conditional \hasns\ and \hasrem\ metrics to include the physical requirement that the probability that a system with a post-merger matter remnant must necessarily
contain a \ac{NS}. Then we consider an extended set of \ac{EOS} and marginalize the results over this set to minimize possible systematics arising from the \ac{EOS}.
Finally, we perform a thorough study of alternative \ac{ML} algorithms and compare their performance on the synthetic signals in \ac{O2} data and confident \ac{O3}
\ac{GW} detections from the \ac{GWTC3} catalog.

The structure of the paper is as follows. In Sect.~\ref{algos}, we introduce the classification algorithms. In Sect.~\ref{dataset} we describe the data set and the
classification schema. Section \ref{results} is dedicated to the results of each method and their comparison. Conclusions and future developments are presented in
Sect.~\ref{conclusions}.
