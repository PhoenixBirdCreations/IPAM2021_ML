\section{Introduction}
In our introduction we want to start from big picture (LIGO, ML, pipelines)
and narrow down to what work we are presenting here and why it is important. We
also describe in which ways it is novel and how it compares to previous works
like in~\cite{Chatterjee:2019avs}. We may also want to cite~\cite{Sachdev:2020lfd}. 


The breakthrough observation of the binary neutron star (BNS) merger simultaneously in gravitational and electromagnetic (E/M) waves \cite{LIGOScientific:2017ync} marked a new era of multi-messenger astronomy including for first time gravitational waves (GWs). The event GW170817 gave answers to open questions, such as the origin and production of Gamma-ray bursts (*cite), the production of heavy elements during a merger of BNS (*cite), rule out Equations of States (*cite) and alternative theories of gravity (*cite), +?. Among the challenges that it brings is the early alert of the E/M telescopes. The sources that can produce an E/M counterpart are the coalescences of a neutron star with another neutron star (NSNS) or with a black hole (NSBH). In the case of NSNS: GRBs, neutron star. For NSBH: accretion disk --> short GRBs.

- Mention empirical fits, Foucart?

- Real time inference: What was done in O2, O3 (Deep)

- Probabilities HasNS, HasRemnant

- What we suggest, Random Forest classification (what is it, what are the advantages)

In this work, we revisit and extend the problem of classification in low-latency. First, we propose the use of a combined probability of whether there is a NS in the event and whether there is a remnant. By doing so, we exclude the unphysical situation of having a larger probability of having a remnant than having a NS (like in *ref). In addition, we perform a thorough study of different algorithms, perform several tests and account for different EoSs. We povide a direct comparison between the methods we used, as well as the ones already implemented e.g. in Deep *ref. 
