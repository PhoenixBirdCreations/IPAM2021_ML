\section{Introduction}
%In our introduction we want to start from big picture (LIGO, ML, pipelines)
%and narrow down to what work we are presenting here and why it is important. We
%also describe in which ways it is novel and how it compares to previous works
%like in~\cite{Chatterjee:2019avs}. We may also want to cite~\cite{Sachdev:2020lfd}. 

The breakthrough \mmt{\sout{observation} detection of \sout{the binary neutron star (BNS) merger simultaneously in gravitational and electromagnetic (EM) waves} the gravitational waves (GWs) generated by a binary neutron star merger (BNS) together with an electromagnetic (EM) counterpart}~\cite{LIGOScientific:2017ync} marked a new era of multi-messenger astronomy \mmt{\sout{including for first time gravitational waves (GWs)}. So far, the LIGO-Virgo-KAGRA (LVK) Collaboration~\citep{LIGO:2015,Virgo:2015,KAGRA:2019} has reported the observation of such merger, GW170817~\cite{Abbott:2017_GW170817} with its EM counterpart GRB170817A/AT2017gfo~\cite{grb,Abbott:2017b}, and also GW190425 \cite{Abbott:2020}, but in this last case no EM counterpart was detected.} The event GW170817 gave answers to open questions, such as the origin and production of gamma-ray bursts (*cite), the production of heavy elements during a merger of BNS (*cite), rule out Equations of State (EoS) (*cite) and alternative theories of gravity (*cite), +?. Such simultaneous observations are crucial for a variety of scientific fields, like astrophysics, particle physics, cosmology, and therefore the necessity to coordinate them successfully emerges. Among the challenges that this attempt brings is the early alert of the EM telescopes/facilities from the GW detectors. Some of the expected sources to produce an EM counterpart are the \mmt{\sout{coalescences of binary compact objects (CBCO)} compact binary coalescences (CBC)}, such as a neutron star with another neutron star (NSNS) or with a black hole (NSBH). In the case of a NSNS coalescence: GRBs, neutron star. For NSBH: accretion disk --> short GRBs. \mmt{[MMT: need to develop this more.]}


Numerical relativity (NR) simulations can provide accurate information about the coalesence of the binary system and its aftermath/remnant. However, these simulations are computationally expensive and require a lot of time, making them inapropriate to alert follow-up EM searches. It is important to alert the EM ground and space telescopes quickly and accurately. (For example in the event of 2017 the EM telescopes were informed after ... *ref?)
   
%- Mention empirical fits, Foucart? 
 
In the previous observing runs in LIGO and Virgo there were used different matched-filtering pipelines in low latency in order to detect GWs from CBC, such as the GstLAL-based inspiral pipeline (GstLAL) ~\cite{Sachdev:2020lfd}, the PyCBCLive ~\cite{Nitz_2018}, the MBTAOnline ~\cite{Adams_2016}.+? During the second observing run O2 it was the first time that real-time data was made available to EM observatories to support follow-up observations. Classification of the binary systems was performed according to the following probabilities: the probability that at least one of the objects in the binary is a NS, p(HasNS) and the probability that there will be a matter remnant p(HasRemnant). For instance, in a BNS merger both probabilities will be equal to one, while in a binary black hole (BBH) merger, both probabilities will be zero. In ~\cite{Chatterjee:2019avs} the authors approach the problem as binary classification and develop a new method based on supervised learning. 

%Say a few things about the early recovery of the parameters and explain what Deep does.

In this work, we revisit and extend the problem of classification in low-latency. First, we propose the use of a combined probability of whether there is a NS in the event and whether there is a remnant. By doing so, we exclude the unphysical situation of having a larger probability of having a remnant than having a NS (like in *ref??). In addition, we perform a thorough study of different algorithms, perform several tests and account for different EoSs. We povide a direct comparison between the methods we used, as well as the ones already implemented e.g. in Deep *ref. 

Different methods of classification were considered, such as ... and were tested for the same data. Here, we present the ones that performed best, the Random Forest (RF) and KNeighborClassifier (KNN). The last one is also used in ~\cite{Chatterjee:2019avs} and thus we provide a direct comparison. 

The structure of the paper is as follows. In Section 2 there is a detailed description of the data used, as well as the labeling introduced. In Section 3, the classification methods, namely Random Forest and KNN are presented. +   
