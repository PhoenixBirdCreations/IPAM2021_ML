\section{Dataset and labeling\label{dataset}}
\label{sec:dataset}
In this section, we discuss the data conditioning process in our study. The data we use for training and testing are the same injections during O2 as in  [REF Deep's paper], also aiming to predictict the probability of a detected event of having a neutron star (HasNS) and/or a remnant object (HasREM) that could emit an electromagnetic counterpart. As we utilize supervised ML techniques, we first label the data accordingly.

HasNS is considered true if $m2_{inj}< 3M\odot$ as in [REF]. For HasREM and following [REF] we use the Focault formula for the remnant mass and if the result is greater than 0 we label the event as true . However, this formula depends on the Equation of State (EoS) as it utilizes the compactness of the NS. The injections used in our study utilize the 2H EoS, but we relabel the data with a different one as explained later, as the EoS used for labelling the training data will affect the ability of predicting correctly a real event.

Predicting the presence of a NS and the creation of a remnant object like this requires a binary classifier for each of the two tasks. We employ a new approach that utilizes the relationship between HasNS and HasREM to train a single multilabel classifier. We relabel the data into 3 mutually exclusive categories: label 0 if there is no NS and no remnant, label 1 if there is a NS but no remnant, and label 2 if there are both. Our labelling is summarized in table \ref{tab:labels}. This labeling eliminates the possibility of an unphysical classification of the event, where $p(HasREM)>p(HasNS)$, as there is no category for HasREM but no NS.

As the categories are mutually exclusive, $p(0)+p(1)+p(2)=1$. Therefore, p(HasREM)=p(2) and for the NS, p(hasNS)=p(hasNS and hasREM $\cup$ hasNS and No hasREM)=p(2$\cup$1)=p(2)+p(1)=1-p(0). This approach allows us to use a single classifier while avoiding any unphysical classifications without further treatment of the data or the output.\textcolor{blue}{In this work, RF and KNN were trained on data labeled as in \ref{tab:labels}, however, genetic programming algorithm used in this work was trained on binary labels for hasNS and hasRemnant separately because of its limitations in performing multi-label classification}.

\begin{table}[h]
\centering
\begin{tabular}{@{}ccc@{}}
\toprule
HasNS & HasRem & Our label \\ \midrule
0     & 0      & 0         \\
1     & 0      & 1         \\
1     & 1      & 2         \\ \bottomrule
\end{tabular}
\caption{Labelling adopted for classification of having a NS and having a remnant with the same classifier}
\label{tab:labels}
\end{table}

In order to avoid conditioning the results of HasREM too heavily to the equation of state (EOS) selected for training, we labeled the dataset with the 23 EOS selected by the LIGO-Virgo collaboration in [ref] for studies of neutron stars. These EOS cover a wide parameter space, making together a robust estimation for the neutron star composition.

\textcolor{blue}{Each algorithm presented here} was trained for each of the 23 EOS and predictions were obtained. The final prediction provided is a weighted average between the results of all of them, with the weights determined by the Bayes factor of each EoS, as explained in [ref2]. This approach allows for the consideration of multiple EOS and their relative likelihoods, resulting in a more robust and accurate prediction.

