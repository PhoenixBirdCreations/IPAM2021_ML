\section{Data Set} \label{dataset}

In our analysis, we employ the same data set as Ref.~\cite{Chatterjee:2019avs} This allows us to directly compare the performance of the various algorithms and the new labeling
scheme to the performance of the \ac{KNN} algorithm, which was deployed in the latest \ac{LVK} observing run. The data set is described in detail in Ref. \todo{chatterjee and messick} \mmt{[MMT: which paper is this?]}. Only its most relevant features are presented below.

We train the \ac{KNN}, \ac{RF}, and \ac{GP} algorithms on \todo{XX} days of data \mmt{[MMT: Sushant?]} from \ac{LVK}'s \ac{O2} observing run with \todo{XX} synthetic \ac{BNS}, \ac{NSBH}, and \ac{BBH} signals
injected into it \todo{briefly how the data set was built (replay?) and what's was its purpose (MDC? tests?)}. \todo{Now explain briefly the characteristics of the signals, for example
their mass range, spin properties, FAR\dots} The \ac{CBC} signals are recovered with one of the standard search \ac{LVK} pipelines \todo{used in\dots deployed for\dots?}, the GstLAL
pipeline \todo{ref}. \todo{A brief sentence to say how it works? e.g., match-filter?} The dimension of the data set is determined by the number of signals recovered by the pipeline after
proper selection cuts are applied. Each event is identified by a \todo{X}-dimensional state vector that contains the parameters of the signal recovered by GstLAL and an additional set of
quantities derived from these parameters. The first \todo{X} elements of the state vector are the two component masses, \todo{the spins\dots complete this} from GstLAL's best matching
template and the \ac{FAR} of the event. In this work, we restrict our analysis to signals recovered with a \ac{FAR} of \todo{XX}. \todo{Explain why.} The remaining elements of the state
vector are \todo{list of other derived quantities\dots} 


\todo{For both algorithms we use events with 5 features: the two masses, their corresponding spins and the SNR of the detection.}
