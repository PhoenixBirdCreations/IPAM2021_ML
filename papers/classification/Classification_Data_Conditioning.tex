\subsection{Data Set} \label{dataset}

We use a large data set $D$ of simulated \ac{BNS}, \ac{NSBH}, and \ac{BBH} events that was first used for the space-time volume sensitivity analysis of the \ac{LVK} GstLAL search
\cite{Sachdev:2019vvd,PhysRevD.95.042001,Sachdev:2020lfd} and later employed in Ref.~\cite{Chatterjee:2019avs}. This allows us to directly compare the performance of the various algorithms and the new
labeling scheme to the performance of the \ac{KNN} algorithm that is deployed in the current \ac{LVK} observing run. 

The simulated signals are coherently injected in two-detector data from the \ac{O2} \ac{LVK} observing run. The injection population is built with uniform/loguniform distribution of component masses whereas the component spins are aligned and injected according to isotropic distributions. Further details on the waveforms and injection parameters can be found in Ref.~\cite{Chatterjee:2019avs}. The data set $D$ includes approximately 200,000 injected signals that are recovered by the GstLAL pipeline with a \ac{FAR} $\le$ 1/month. The \ac{RF} and \ac{KNN} algorithms are trained and tested on the injected and recovered intrinsic source properties (primary and secondary masses and spins) and on the recovered \ac{SNR}. 

