\subsection*{Genetic Programming}

\ac{GP} is a supervised evolutionary algorithm \cite{koza_genetic} that relies on genetic operations inspired by natural selection, a fitness function, and multiple generations of
programs to perform a user-defined task \cite{GP_book}. Programs with a higher fitness score are more likely to be passed down to future generations. As the population of programs evolves,
the ability to solve the task at hand increases \cite{Kai_thesis}. \ac{GP} can be used in regression or classification problems to discover mathematical relationships between
variables or data. 

KarooGP \cite{KarooGP,KarooPYPI}, an open-source Python code, is used in this work. The output of the algorithm is a multivariate mathematical expression in the form of a human-readable
syntax tree, where the nodes are mathematical operators and the leaves are input variables. This aspect makes the method very suitable for the post-processing application of large data
sets, as the evolutionary process can be reviewed at each step \cite{Cavaglia_2020}. The algorithm is initialized through a stochastic population of trees, which is left to evolve according
to pre-defined genetic operators such as reproduction, mutation, and crossover. At each generation, the performance of each individual tree in solving the problem is evaluated through a
fitness score. A random subset of the best-performing trees is carried forward to the next generation. The process repeats until a set of user-defined conditions are met. Parameters such as
the initial population size, tree depth, tournament size for competing trees, number of generations, and termination criterion can be tuned for better algorithmic performance.
\todo{citation}.
