\subsection*{Genetic Programming}

\ac{GP} is a supervised evolutionary algorithm \cite{Koza1992} that relies on genetic operations inspired by natural selection, a fitness function, and multiple generations of
programs to perform a user-defined task. Programs with a higher fitness score are more likely to be passed down to future generations. As the population of programs
evolves, the ability to solve the task at hand increases \cite{StaatsThesis}. \ac{GP} can be used in regression or classification problems to discover mathematical relationships
between variables or data. 

In this work, we use the KarooGP open-source Python \ac{RF} implementation of \ac{GP} \cite{KarooGP,KarooPYPI}. The output of KarooGP is a multivariate mathematical expression
in the form of a human-readable syntax tree, where the nodes are mathematical operators and the leaves are input variables. The algorithm is initialized through a stochastic
population of trees, which is left to evolve according to pre-defined genetic operators such as reproduction, mutation, and crossover. At each generation, the performance of each
individual tree in solving the problem is evaluated through a fitness score. A random subset of the best-performing trees is carried forward to the next generation. The process
repeats until a set of user-defined conditions are met. Parameters such as the initial population size, tree depth, tournament size for competing trees, number of generations, and
termination criterion can be tuned for better algorithmic performance. Due to the stochastic nature of the \ac{GP} algorithm, the training process is typically repeated a number of times to obtain distinct multivariate expressions, which are then independently used to classify the data. 
