\section{Random Forest (RF)}

A RF is an ensemble of decision trees. One of its major strengths is that every train trains and classifies independently from the rest, while the RF classification joints all the results and assign as category the mode from the trees. The probability of belonging to a category is therefore straightforward, being the number of trees that chose it divided by the total number of trees. Notice that the training and evaluation of a RF can be accelerated by parallelization, as computations inside each tree are independent from the rest.

The training of a RF is usually done with bootstrap, a technique that assigns a random subset of the training dataset to each tree. This prevents overfitting as every individual classifier is not exposed to the same data, and encourages pattern recognition by studying the same data from different subsets. Every decision tree is composed by nodes, where data its splitted until the different categories are separated. At each node, a subset of the features of the data is selected along threshold values that maximize the information gain at the separation. The binary splitting at each node gives the tree its name, as it can be visualized as roots going deeper at separations.

We use the RK implementation in scikitlearn [REF AND DETAILS]. The main hyperparameters to tune in this module are the number of trees, the maximum depth allowed and the information gain criteria used at splitting (two are offered). We have observed that the maximum number of features to be considered in a node can be kept fix as the square root of the total number of features. Given that the aiming of this work is to improve the current low latency classification, the model once trained can occupy a restricted amount of memory. Therefore before searching the optimum hyperparameters for our dataset, we restrict those which make heavier models: the number of trees and their depth. We set to 100 the maximum number of trees the forest may have, and 25 their possible maximum depth.

For the RF we use events with 5 features: the two masses, their corresponding spins and the SNR of the detection. In the tuning of the hyperparameters we measure the performance  by its score: the number of events correctly classified against the total number of events in the testing dataset, if threshold is taken as 0.5. As all categories are balanced, this approach is enough to roughly compare models. The best model found achieves a score of $whatever$ using $number$ of trees, $number$ maximum depth and $name$ criteria for the information gain.
