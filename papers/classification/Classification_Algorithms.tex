\section{Classification Algorithms\label{algos}}
\subsection*{\mmt{K-Nearest Neighbors}}

\todo{This section needs to be rewritten and expanded. Also, citations must be added. It should start with a brief sentence of what KNN is, then how it works, then what is our implementation. See the GP section ofr inspiration.}

\mmt{The K-Nearest Neighbors (KNN) is a non-parametric supervised algorithm~\cite{Fix:1951,Cover:1967} that relies on the fact that similar things are near to each other. It captures the idea of similarity by computing the distance between points, and it can be applied in classification problems~\cite{Guo:2004}, being renamed as KNeighborClassifier.}

\mmt{The KNeighborClassifier algorithm works as follows: the algorithm computes the distance between each point of data and the points of a training dataset, with a chosen metric. Then, it sorts the points in ascending order vased on the distance to the testing point. By choosing the top $K$ neighbors (training points) from the sorted array, KNN assigns a label to the data point, which is the most frequent label of the chosen neighbors.}

\mmt{In this work, we are going to use the KNeighborClassifier implementation of scikit-learn~\cite{Pedregosa:2011}, an open-source Python code. The algorithm is initialized through different parameters that should be tuned in order to optimize its performance. The number of neighbors is fixed by the user, and its optimal value turns out to be $K \approx 2n+1$~\cite{Chatterjee:2019avs}, where $n$ is the number of features that each datapoint has. The distance between the different neighbors can be computed with different metrics -euclidean, manhattan, chebysev, mahalanobis-, and the prediction can be weighted by the inverse of the distance or all the points in each neighborhood can be weighted equally. To compute the nearest neighbors, several algorithms are available, such as the BallTree, KDTree or a brute-force search. These are the parameters that we will tune in this work in order to optimize the algorithm for our concrete problem. We will apply cross validation by computing the score, i.e. the fraction of correctly classified events, over a range of possible values for the parameters of the algorithm.} 


\subsection{Random Forest (RF)}

A RF is an ensemble of decision trees. One of its major strengths is that every train trains and classifies independently from the rest, while the RF classification joints all the results and assign as category the mode from the trees. The probability of belonging to a category is therefore straightforward, being the number of trees that chose it divided by the total number of trees. Notice that the training and evaluation of a RF can be accelerated by parallelization, as computations inside each tree are independent from the rest.

The training of a RF is usually done with bootstrap, a technique that assigns a random subset of the training dataset to each tree. This prevents overfitting as every individual classifier is not exposed to the same data, and encourages pattern recognition by studying the same data from different subsets. Every decision tree is composed by nodes, where data its splitted until the different categories are separated. At each node, a subset of the features of the data is selected along threshold values that maximize the information gain at the separation. The binary splitting at each node gives the tree its name, as it can be visualized as roots going deeper at separations.

We use the RK implementation in scikitlearn \citep{Pedregosa:2011}. The main hyperparameters to tune in this module are the number of trees, the maximum depth allowed and the information gain criteria used at splitting (two are offered). We have observed that the maximum number of features to be considered in a node can be kept fix as the square root of the total number of features. Given that the aiming of this work is to improve the current low latency classification, the model once trained can occupy a restricted amount of memory. Therefore before searching the optimum hyperparameters for our dataset, we restrict those which make heavier models: the number of trees and their depth. We set to 300 the maximum number of trees the forest may have, and 45 their possible maximum depth.

For the RF we use events with 5 features: the two masses, their corresponding spins and the SNR of the detection. In the tuning of the hyperparameters we measure the performance  by its score: the number of events correctly classified against the total number of events in the testing dataset, if threshold is taken as 0.5. As all categories are balanced, this approach is enough to roughly compare models. 




\subsection*{Genetic Programming}

\ac{GP} is a supervised evolutionary algorithm \cite{koza_genetic} that relies on genetic operations inspired by natural selection, a fitness function, and multiple generations of
programs to perform a user-defined task \cite{GP_book}. Programs with a higher fitness score are more likely to be passed down to future generations. As the population of programs evolves,
the ability to solve the task at hand increases \cite{Kai_thesis}. \ac{GP} can be used in regression or classification problems to discover mathematical relationships between
variables or data. 

KarooGP \cite{KarooGP,KarooPYPI}, an open-source Python code, is used in this work. The output of the algorithm is a multivariate mathematical expression in the form of a human-readable
syntax tree, where the nodes are mathematical operators and the leaves are input variables. This aspect makes the method very suitable for the post-processing application of large data
sets, as the evolutionary process can be reviewed at each step \cite{Cavaglia_2020}. The algorithm is initialized through a stochastic population of trees, which is left to evolve according
to pre-defined genetic operators such as reproduction, mutation, and crossover. At each generation, the performance of each individual tree in solving the problem is evaluated through a
fitness score. A random subset of the best-performing trees is carried forward to the next generation. The process repeats until a set of user-defined conditions are met. Parameters such as
the initial population size, tree depth, tournament size for competing trees, number of generations, and termination criterion can be tuned for better algorithmic performance.
\todo{citation}.


In order to decide which method gives a better performance in classifying this kind of events, we can apply them over testing data and finally do a comparison between both. A way to see how data is classified we can construct histograms where the number of events that are classified with a label (\texttt{HasNS/HasRemnant}) \texttt{True} or \texttt{False} will change with a given threshold of the probability. For an algorithm with perfect performance, all the events with label \texttt{True (False)} should be at \textit{p}(\texttt{label}) = 1 (\textit{p}(\texttt{label}) = 0).

Another way to check the algorithm's performance is by building the so-called \textit{Receiver Operating Characteristic (ROC) Curve}. They show the variation of the true-positive rate (or efficiency) with the false-positive rate given a certain threshold for the probability. An algorithm with a proper performance will give a steeper ROC curve, or in other words, will have a higher eficiency with a lower false-positive rate.  


In the ROC curves that we will present in the following subsections, we highlight three reference EoS in color, from which we show results in more detail. We select BHF\_BBB2 because is the model that give the lowest maximum mass, MS1\_PP as the model with the bigger maximum mass, and we also include SLy because is the most accepted EoS for NS modeling (reference), and is the one that was used in the injections that are our dataset.

Here are the results for the methods:

\subsection{KNN Results}

%\mmt{We are only using 5 features, the independent variables: 
%\begin{equation*}
	%\big[m_1,m_2,\chi_1,\chi_2, \rm{SNR}\big]\,.
%\end{equation*}}

\subsubsection{\mmt{Has NS}}
\mmt{The metric we use to compute the distance between neighbors is the \textit{Manhattan} metric (or the Minkowski's $L1$ distance),  which is the distance between two points measured along axes at right angles. Having $p_1(x_1,y_1)$ and $p_2(x_2,y_2)$ the distance will be}
\begin{equation}
	d = |x_1-x_2|+|y_1-y_2|\,.
\end{equation}

\mmt{Moreover, the points are weighted uniformly.  After applying cross-validation,  we get that the optimal number of neighbors is $K_{\rm NS} = 10$, with a mean score $\rm{s_m} = 0:9718355224352762$ and a testing score  $\rm{s_t} = 0.9723828730478842$.} 

%\begin{figure}
%    \includegraphics[width = 0.4\textwidth]{CrossValK.pdf}
%    \caption{Score of our KNN model as a function of the number of neighbors. We are considering \textit{HasNS}.}
%    \label{fig:crossvalK}
%\end{figure}
    
\begin{figure}
    \includegraphics[width=0.45\textwidth]{figs/conf_matrix_NS.pdf}
    \caption{Confusion matrix for our model for \textit{HasNS}, using the independent recovered values. }
    \label{fig:confmat}
\end{figure}

\begin{figure}
    \includegraphics[width = 0.4\textwidth]{plot_fig4_chatt_spins.png}
  %   \includegraphics[width = 0.4\textwidth]{/Users/miquelmiravet/Projects/IPAM_LA/ML_group/IPAM2021_ML/algo/classy_KNN/PLOTS_KNN/NS_set/plots_miq/plot_fig4_chatt_snr.png}
    \caption{Probability of having a remnant as a function of the values of the masses. The different panels show the results for different spins. The solid red line depicts the threshold mass for $m_2$.}
    \label{fig:m1m2}
\end{figure}

\begin{figure}
	\includegraphics[width =0.4\textwidth]{ROCplot.pdf}
    \caption{Relation of the true and false positive rates as a function of the threshold applied to make the decision between having or not having a remnant. }
    \label{fig:roc}
\end{figure}

\mmt{In Fig. you can find how the mean score changes with the number of neighbors of the algorithm.  The confusion matrix appears in Fig.~\ref{fig:confmat}, the probability as a function of $m_1$ and $m_2$  is shown in Figs.~\ref{fig:m1m2}, and the true and false positive rates in terms of the threshold probability appear in Fig.~\ref{fig:roc}.}


 %plots and comments
\subsection{RF Results}

\begin{figure}
\centering
\includegraphics[width=0.9\textwidth]{/figs/BHF_BBB2_NShist}
\includegraphics[width=0.9\textwidth]{/figs/BHF_BBB2_REMhist}
\caption{\label{fig:RF_hist_BHFBBB2}}
\end{figure}
 %plots and comments
\input{GP_Results.tex}


\subsection{Algorithm comparison}
