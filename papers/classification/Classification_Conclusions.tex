\section{Conclusions\label{conclusions}}

TO DO LIST:
\\
- Brief summary of the problem we are facing: we need rapid identification of the components of the binary merger, we apply ML to do so. \\
- Say what we did: tested and compared the algorithms on O2, with the scores. Then we come with the Bayesian probs and we make some fits to quickly get the probabilities when there is a real time event. \\
- Discuss the results (MOST IMPORTANT THING): tested in O2, the algorithms perform better for \hasrem\, and RF does it better than KNN. When applied on O3 (this applies for both the scores and the Bayesian probs), the algorithms are better on \hasns\ (and RF still outperforms), but they get worse at \hasrem\, and even KNN is slowly better in this case. It seems that Sushant's assumption is right, and RF finds it more difficult to classify different pipelines and new events (O3 instead of O2) -> it is less adaptable. \\
- This is already mentioned in the results, but state again that RF applies hard cuts on the equations of state, whereas KNN relies more on the data surrounding the event of interest. This makes the probabilities from both algorithms to be different when it comes to classifying events located around the mass gap, e.g. GW190824. 
-Also mentioned in the results: the differences in the parameter sweeps. RF applies a hard cut for large primary masses, and that is why the probability in that region for \hasns\ is so uniform, as opposed to KNN, which gives a noisier parameter sweep, because it only needs few neighbors that are labelled differently to provide a higher probability in a certain region.  \\
- Possible future applications: use a training dataset that includes all the pipelines to improve the performance of RF on events provided by pipelines different from GstLAL; provide more data located in the mass gap to improve KNN on the classification of events located in that region. 
