\begin{abstract}

Because of the electromagnetic radiation produced during the merger, compact binary coalescences with neutron stars may result in multi-messenger observations. In order to follow up on the
gravitational-wave signal with electromagnetic telescopes, it is critical to promptly identify the properties of these sources.  This identification must rely on the properties of the progenitor
source, such as the component masses and spins, as determined by low-latency detection pipelines in real time. The output of these pipelines, however, might be biased, which could decrease the
accuracy of parameter recovery. Machine learning algorithms are used to correct this bias. In this work, we revisit this problem and discuss the implementation of two supervised machine learning
algorithms, K-Nearest Neighbors and Random Forest, which are able to predict the presence of a neutron star and post-merger matter remnant in the compact binary coalescence more accurately than the
low-latency infrastructure currently used by LIGO, Virgo, and KAGRA. Additionally, we present a novel approach for calculating the Bayesian probabilities for these two metrics. Instead of metric
scores derived from binary machine learning classifiers, our scheme is designed to provide the astronomy community well-defined probabilities. This would deliver a more direct and easily
interpretable product to assist electromagnetic telescopes in deciding whether to follow up on gravitational-wave events in real time.

\end{abstract}
