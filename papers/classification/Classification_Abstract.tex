\begin{abstract}
  The detection of gravitational-wave signals generated by compact binary
    coalescences that involve neutron stars can lead to multi-messenger
    observations, due to the electromagnetic radiation that is produced after
    merger. A rapid identification of the binary elements is necessary to
    perform proper follow-up operations with electromagnetic telescopes. This
    identification is done by looking at the properties of the binary system,
    such as the masses and spins, inferred by low latency pipelines. However,
    the resulting predictions are dominated by systematic errors that might
    lead to a misidentification of the system. In this work, we compare two
    supervised machine learning algorithms, K-Nearest Neighbors and Random
    Forest, that succeed in reducing such errors. We also present a new way to
    compute the probabilities of having a neutron star in the system and of
    producing an electromagnetic counterpart with a Bayesian approach. This
    will allow the astronomy community to receive physical probabilities
    instead of those coming from binary classification scores, resulting in
    more direct and easily interpretable information that will help them decide
    whether to follow up on gravitational-wave events with electromagnetic observatories. 
\end{abstract}
