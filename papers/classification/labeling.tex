\subsection{Labeling Scheme}\label{labeling}

%Each event must be classified based on its \hasns\ and \hasrem\ properties. These properties depend on which \ac{EOS} is used to describe the matter present in the coalescing binary
%system. Throughout the paper, we consider a set of \tocheck{23} different \ac{EOS} that span a wide range of models \todo{Explain what EOS we consider or give refs.} Therefore, we
%produce \tocheck{23} distinct labeled data sets that are separately used for training, testing, and validation. The overall performance of the algorithms is then assessed by marginalizing on the whole set of \ac{EOS}, i.e., by taking a weighted average with weights determined by the Bayes factors of Ref.~\todo{need ref}. Throughout the paper we also show results for three single \ac{EOS} (\tocheck{A, B, and C}) to assess the performance of the \ac{KNN}, \ac{RF}, and \ac{GP} algorithms on physical systems with distinctive properties. The \tocheck{A} \ac{EOS} was used in Ref.~\tocheck{\cite{Chatterjee:2019avs}}. This allow us to directly compare our results with the results of Ref.~\tocheck{\cite{Chatterjee:2019avs}}.

To label the synthetic data set $D=D_R\oplus D_S$, we follow the practice in use in the \ac{LVK} and identify an event with \hasns:\true\ when at least one of the injected component masses is less
than the maximum \ac{NS} mass allowed by the \ac{EOS}. The value of the maximum \ac{NS} mass ranges from $1.922$ $M_{\odot}$ to $2.753$ $M_{\odot}$ across the various \ac{EOS}, %and is equal to 1.922 $M_\odot$, 2.054 $M_\odot$, and 2.753 $M_\odot$, for the {\tt BHF\_BBB2}, {\tt SLy}, and {\tt MS1\_PP} \ac{EOS}, respectively. 
\tocheck{The lowest and largest maximum \ac{NS} masses correspond to the {\tt BHF\_BBB2} and {\tt MS1\_PP} \ac{EOS}, respectively. We will highlight these two cases together with the {\tt SLy} \ac{EOS}, which is the most accepted \ac{EOS} in the astrophysics community.} We set the \hasrem\ label as \hasrem:\true\ and \hasrem:\false\ for
\ac{BNS} and \ac{BBH} systems, respectively. The value of the \hasrem\ label for \ac{NSBH} events depends on the \ac{EOS} of the \ac{NS}. To identify the \hasrem\ event class for \ac{NSBH} systems, 
we follow Ref.~\cite{Chatterjee:2019avs} and apply Eq.~(4) from Ref.~\cite{Foucart:2018rjc}, colloquially known as the \emph{Foucart formula}.

The Foucart formula is an empirical fit that predicts the total mass of the accretion disk, the tidal tail, and the ejected mass from the final \ac{BH}. The main parameters of the fit are
the compactness of the \ac{NS},  the \ac{NSBH} binary system's symmetric mass ratio, and the normalized \ac{ISCO} radius. The tidal disruption of the \ac{NS} is affected by the mass and spin of the \ac{BH}. A highly spinning, low-mass \ac{BH}'s small ISCO allows the \ac{NS} to inspiral closer to the \ac{BH} and its tidal force to tear the \ac{NS} apart, resulting in matter ejecta. If the tidal forces are weak or the \ac{NS} is very compact, the \ac{BH} will swallow the \ac{NS} and there will be no remnant mass.

Different \ac{EOS}s have different thresholds for the mass of the remnant. We label events with inferred masses less than this threshold as \hasrem:\true. For events with component masses between $\sim 2.5$ and $3.5 M_{\odot}$, the stiffness of the \ac{EOS} is the main factor determining the \hasrem\ label.
