\subsection*{Labeling Schema}

%Each event must be classified based on its \hasns\ and \hasrem\ properties. These properties depend on which \ac{EOS} is used to describe the matter present in the coalescing binary
%system. Throughout the paper, we consider a set of \tocheck{23} different \ac{EOS} that span a wide range of models \todo{Explain what EOS we consider or give refs.} Therefore, we
%produce \tocheck{23} distinct labeled data sets that are separately used for training, testing, and validation. The overall performance of the algorithms is then assessed by marginalizing on the whole set of \ac{EOS}, i.e., by taking a weighted average with weights determined by the Bayes factors of Ref.~\todo{need ref}. Throughout the paper we also show results for three single \ac{EOS} (\tocheck{A, B, and C}) to assess the performance of the \ac{KNN}, \ac{RF}, and \ac{GP} algorithms on physical systems with distinctive properties. The \tocheck{A} \ac{EOS} was used in Ref.~\tocheck{\cite{Chatterjee:2019avs}}. This allow us to directly compare our results with the results of Ref.~\tocheck{\cite{Chatterjee:2019avs}}.

To label the synthetic data sets $D$, we follow the practice in use in the \ac{LVK} and identify an event with \hasns:\true\ when at least one of the injected component masses is less than the
maximum \ac{NS} mass allowed by the \ac{EOS}. The value of the maximum \ac{NS} mass ranges from \tocheck{X} to \tocheck{Z} across the various \ac{EOS}, and is equal to
$\tocheck{X}M_\odot$, $\tocheck{Y}M_\odot$, and $\tocheck{Z}M_\odot$, for the \tocheck{A, B, and C} \ac{EOS}, respectively. The \hasrem\ property depends on the \ac{EOS} through the
compactness of the \ac{NS}. To identify the \hasrem\ event class, we follow Ref.\ \cite{Chatterjee:2019avs} and apply the \mmt{Foucart formula (Eq.~(4) from \cite{Foucart:2018rjc})}.

\todo{any volunteer to expand this with more details?}

\mmt{The fit presented in \cite{Foucart:2018rjc} depends, apart from the compactness of the \ac{NS}, on the symmetric mass ratio of the binary system, the normalized ISCO radius and some non-linear effects. It is an empirical fit to predict the total mass of the accretion disk, the tidal tail and the ejected mass from the final BH (in case it is an NSBH merger). Since we expect \hasrem:\true\ for a BNS and \hasrem:\false\ for a BBH, the most interesting scenario to study P(\hasrem\) is the NSBH system. In this case, the mass and spin of the BH affect the tidal disruption of the NS. For example, if the BH is not too massive and it has a high spin, the ISCO orbit will be small and this will allow the NS to inspiral closer to the BH, and the tidal force of the BH will tear the NS apart. Therefore, there will be remnant matter. On the other hand, if the tidal forces are not strong enough, or the NS is compact, the BH will swallow the NS, and there will be no remnant mass. 

Each EOS will give a different threshold value for the mass of the remnant, and inferred masses below this value will be labeled with the \hasrem\ class. As in the case of \hasns\, there will be a region in the parameter space of the component masses (around $\sim 2.5-3.5 M_{\odot}$) where the prediction will change depending on the stiffness of the EOS we consider.}

