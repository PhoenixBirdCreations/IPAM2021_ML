% voodoo for arXiv scripts
\pdfoutput=1
\documentclass[aps,prd,twocolumn,superscriptaddress,preprintnumbers,floatfix,nofootinbib]{revtex4-2}

\usepackage{graphicx}
\usepackage{amsmath}
%\usepackage{mdwlist}
%\usepackage[caption=false]{subfig}
%\usepackage{siunitx}
%\usepackage{placeins}
\usepackage{color}
%\usepackage{standalone}
\usepackage{dcolumn}
\usepackage{tensor}
\usepackage{bm}
\usepackage{makecell}
%\usepackage{MnSymbol}
% LCS: Proper input/output encoding
\usepackage[T1]{fontenc}
\usepackage[utf8]{inputenc}
\usepackage{microtype}
\usepackage{etoolbox}
\usepackage{amssymb}
\usepackage{mathrsfs}
\usepackage{accents}
\usepackage[normalem]{ulem}
\usepackage[table,dvipsnames]{xcolor}
\usepackage[colorlinks,urlcolor=NavyBlue,citecolor=NavyBlue,linkcolor=NavyBlue,pdfusetitle]{hyperref}
\usepackage[all]{hypcap}
\usepackage[inline,shortlabels]{enumitem}
\usepackage{braket}

\usepackage{array}
\usepackage{diagbox}
\usepackage{color}
\usepackage{colortbl}
\usepackage{hhline}
\usepackage{multirow}

\graphicspath{{./}{figs/}}

\newcommand{\nn}{\nonumber}
\newcommand{\NR}{\text{NR}}

%\newcommand{\lorena}[1]{\textcolor{magenta}{#1}}

%%%%%%%%%%%%%%%%%%%%%%%%%%%%%%%%%%%%%%%%%%%%%%%%%%%%%%%%%%%%
%% Author information
%%%%%%%%%%%%%%%%%%%%%%%%%%%%%%%%%%%%%%%%%%%%%%%%%%%%%%%%%%%%


\usepackage{orcidlink}

\newcommand{\MST}{\affiliation{Institute of Multi-messenger Astrophysics and Cosmology \& Physics Department, 
		Missouri University of Science and Technology, Rolla, MO 65409, USA}}
\newcommand{\OleMiss}{\affiliation{Department of Physics and Astronomy,
		University of Mississippi, University, Mississippi 38677, USA}}
\newcommand{\Torino}{\affiliation{Dipartimento di Fisica,
		Universit`a di Torino \& INFN Sezione di Torino, via P. Giuria 1, 10125 Torino, Italy}}
\newcommand{\Tubingen}{\affiliation{Theoretical Astrophysics Department, 
		Eberhard-Karls University of T\"{u}bingen, T\"{u}bingen 72076, Germany}}
\newcommand{\UAB}{\affiliation{Departament de Matem\`{a}tiques,
		Universitat Aut\`{o}noma de Barcelona, 08193 Bellaterra, Spain}}
\newcommand{\UV}{\affiliation{Departament d’Astronomia i Astrof\'{i}sica,
		Universitat de Val\`{e}ncia, Dr. Moliner 50, 46100, Burjassot (Val\`{e}ncia), Spain}}
		
\begin{document}

%Title of paper
\title{Regression}

\author{Simone Albanesi
	\orcidlink{0000-0001-7345-4415}}
\email{simone.albanesi@edu.unito.it}
\Torino
\author{Marina Berbel
	\orcidlink{0000-0001-6345-1798}}
\email{mberbel@mat.uab.cat}
\UAB
\author{Marco Cavagli\`{a}
	\orcidlink{0000-0002-3835-6729}}
\email{cavagliam@mst.edu}
\MST
\author{Lorena \surname{Magaña Zertuche}
	\orcidlink{0000-0003-1888-9904}}
\email{lmaganaz@go.olemiss.edu}
\OleMiss
\author{Miquel Miravet-Ten\'{e}s
	\orcidlink{0000-0002-8766-1156}}
\email{miquel.miravet@uv.es}
\UV
\author{Dimitra Tseneklidou
	\orcidlink{0000-0003-2582-1705}}
\email{dimitra.tseneklidou@uni-tuebingen.de}
\Tubingen
\author{Yanyan Zheng
	\orcidlink{0000-0002-5432-1331}}
\email{zytfc@umsystem.edu}
\MST
\author{Any other?}
%	\orcidlink{}}
%\email{}


% Because hyperref only gets the *last* author, we need to be explicit.
\hypersetup{pdfauthor={LastName et al.}}

\date{\today}

\begin{abstract}
  Abstract goes here.
\end{abstract}

%\maketitle must follow title, authors, abstract, \pacs, and \keywords
\maketitle

%\tableofcontents

% body of paper here - Use proper  commands
% References should be done using the \cite, \ref, and \label commands

%\tableofcontents

\section{Introduction}
In our introduction we want to start from big picture (LIGO, ML, pipelines)
and narrow down to what work we are presenting here and why it is important. We
also describe in which ways it is novel and how it compares to previous works
like in~\cite{Chatterjee:2019avs}. We may also want to cite~\cite{Sachdev:2020lfd}. 

\section{Improving Recovered Parameters through Regression}
\label{sec:ML_concept}
In this section we may want to expand on the idea of using regression itself.
We should mention which algorithms we attempted to use and why they did not
work. Then we can talk more in detail about the ones that did work in the
following subsections.
\footnote{We may want to add a footnote here and there.}

\subsection{Gaussian Process Regression (GPR)}

\subsection{Neural Networks (NNs)}


\section{Data Conditioning}
\label{sec:BMSFramesIntro}
We should in principle condition the data in the same way but any differences
can also be described here.


%\begin{figure}[t]
%	\centering
%	\includegraphics[width=.45\textwidth]{image_name}
%	\caption{%
%		Here we explain our plot.}
%	\label{image_label}
%\end{figure}

\section{Results}
Here we talk about overall results and specifically from each algorithm in the
subsections below.
\subsection{GPR}

\subsection{NN}


\section{Conclusions}
Reiterate why what we did is important and how it improves current knowledge.

\section*{Acknowledgments}
We thank Deep Chatterjee for useful discussions and for sharing his work, which
helped us compare our results to those of~\cite{Chatterjee:2019avs}. We also
thank Shaon Ghosh for useful discussions and X, Y, and Z for reviewing an
earlier version of this manuscript. Part of this research was performed while
the authors were visiting the Institute of Pure and Applied Mathematics (IPAM), 
University of California Los-Angeles (UCLA). The authors would like to thank 
IPAM, UCLA and the National Science Foundation through grant DMS-1925919 for
their warm hospitality during the fall of 2021. 
%

The authors are grateful for computational resources provided by the LIGO 
Laboratory and supported by the U.S. National Science Foundation Grants 
PHY-0757058 and PHY-0823459, as well as resources from the Gravitational Wave
Open Science Center, a service of the LIGO Laboratory, the LIGO Scientific 
Collaboration and the Virgo Collaboration.
%

The work of L.M.Z. was partially supported by the MSSGC Graduate
Research Fellowship, awarded through the NASA Cooperative Agreement
80NSSC20M0101. 
%
The work of X.Y. was partially supported by NSF Grant No.~PHY-20XXXXX.
%

All plots were made using the python package \texttt{matplotlib}~\cite{Hunter:2007ouj}.

This manuscript has been assigned LIGO Document Control Center number LIGO-P22XXXXX.

\def\bibsection{\section*{References}}

\bibliography{regression}

\end{document}

% Local Variables:
% mode: latex
% End:
